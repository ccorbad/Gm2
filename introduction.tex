\section{Introduction}

The Standard Model (SM) of particle physics, is a very succesfull and predictive theory of fundamental particles and forces of Nature. Recently ATLAS and CMS experiments discovered a new particle with approximate mass of 125.6 GeV \cite{Aad:2012tfa,Chatrchyan:2012ufa}, that is consistent with the SM Higgs Boson. This discovery realize the mechanism behind the Electroweak Symmetry Breaking (EWSB) and the nature of fermion masses through Yukawa couplings of fermion fields with the Higgs boson. Despite that this new particle discovery is an important goal from the theoretical point of view, there still exists problems about the hierarchy of the fundamental interactions and particle masses. This fact and the lack of compatibility of Gravitation with SM has driven to the High Energy Physics (HEP) community to belive that the SM is an effective theory of a more fundamental one.

In order to walk to a more general theory Gravitation, many efforts has been made in the past century. One of the extensions is the Cartan-Einstein Formalism (CEF) (also known as first order formalism) in which the independece between the connection and metric are assumed at the beginning. This theory have no restriction about the symmetric aspect of the connection and contains an antisymmetric part called {\it{torsion}}. Including torsion in a pure gravity action, leads to the same results as in Einstein-Hilbert theory of Gravitation but, including fermionic matter into the picture, gives rises naturally to a four fermion interaction from the equations of motion (see \cite{Hehl:1976kj}). This effect disagrees with the prediction of General Relativity (GR) coupled with fermions and the experimental data will discern if CEF is the correct description of Gravitation.

In $4$-dimensional spacetime, this four fermion interaction is highly suppressed by the inverse of the squared $4$-dimensional Planck mass ($M_{\text{Pl}}\sim10^{18}[\text{GeV}]$). This fact constraint the effect of torsion to be negligible in $4$ dimensions and complicated the measure and distinction of the correct formalism. Extra dimensions models \cite{Randall:1999ee,ArkaniHamed:1998rs} offer an scenario without this hierarchy between Gravitational and SM interaction in the bulk. The main difference between the couplings of the interactions and its hierarchy can be understood as a geometrical nature of the spacetime, relaxing the arbitrariness of the Yukawa's couplings in the SM.

Our aim in this article is to exploit this four fermion interaction and explore one-loop observables using an effective theory coming from extra dimensions. This observables can be constrained using data of the precision tests of the SM \cite{Altarelli:2004fq,Beringer:1900zz} and limits over parameters of extra dimensions model can be achieved. 

This article is organized as follow: In Sec.~\ref{sec:CEF} a brief introduction to Cartan-Einstein Formalism and notation is shown. In Sec.~\ref{sec:extradim} the model and the effective theory in four dimensions coming from the extra dimensions scenario is performed. In Sec.~\ref{sec:oneloop} one-loop observables and form factor in this model is obtained. In Sec.~\ref{sec:constraints} limits coming from precision tests of the SM will be used to constraint the parameters of extra dimensions model. In Sec.~\ref{sec:conclusions} discussions and conclusions about the results obtained in the present work is done. An appendix in Sec.~\ref{sec:cliff} has been included in order to complement the calculations through this article.