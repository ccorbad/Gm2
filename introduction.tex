\section{Introduction}

The Standard Model (SM) of particle physics is a very successful and predictive theory of fundamental particles and  \pro{interaction} of nature. Recently ATLAS and CMS experiments discovered a new particle with approximate mass of \SI{125.6}{\GeV}, that is consistent with the SM Higgs boson~\cite{Aad:2012tfa,Chatrchyan:2012ufa}. This discovery  \pro{will shed light on} the mechanism behind the Electroweak Symmetry Breaking (EWSB) and the nature of fermion masses. \pro{Although this}  discovery \pro{represents}  an important \pro{success}  from the theoretical point of view, there \pro{are}  problems  \pro{such as} the hierarchy of the fundamental interactions and  the lack of compatibility between Gravitation with SM. These have driven  to belive that the SM is an effective theory of a more fundamental one.

Many efforts have been made in the past century %In order to walk
to find a more general theory gravity. One of the extensions is the Cartan-Einstein Formalism (CEF),  in which the independence between the connection and the metric  is assumed. This theory  contains an antisymmetric part of the connection called \textit{torsion}. Including torsion in a pure gravity action, leads to the same results as in Einstein-Hilbert theory of Gravitation, but including fermionic matter into the picture, gives rise to a four fermion interaction  (see Ref.~\cite{Hehl:1976kj}). This effect  \pro{contrasts} with the prediction of General Relativity (GR) coupled with fermions\pro{, therefore}  the experimental data will discern  \pro{whether} CEF is the correct description of Gravitation.

In four-dimensional spacetime, this four fermion interaction is highly suppressed by the inverse square of the  Planck mass ($M_{\text{Pl}}\sim\si{10^{19}}{GeV}$). This  \pro{suppresses}  the effect of torsion  in four dimensions, and hinders the  distinction between theories . Extra dimensional models  offer an scenario without the hierarchy \pro{problem}~\cite{Randall:1999ee,ArkaniHamed:1998rs}. \uline{The main difference between the couplings of the interactions and its hierarchy can be understood as a geometrical nature of the spacetime, relaxing the arbitrariness of the Yukawa's couplings in the SM.} \textsc{No se entiende!!!!!}

The aim of  this article is to  \pro{take advantage of} this four fermion interaction and explore one-loop observables using an effective theory coming from extra dimensions. These observables can be constrained using data of the precision tests of the SM~\cite{Altarelli:2004fq,Beringer:1900zz}, and limits over parameters of extra dimensions model can be achieved. 

The article is organized as follow: In Sec.~\ref{sec:CEF}  the Cartan-Einstein Formalism \pro{is introduced} and notation is \pro{fixed} . In Sec.~\ref{sec:extradim} the model \pro{is presented} and the effective theory in four-dimensions \pro{obtained/performed} . In Sec.~\ref{sec:oneloop} the one-loop observables and form factor of this model are obtained. In Sec.~\ref{sec:constraints} limits coming from precision tests of the SM will be used to constraint the parameters of extra dimensions model. \pro{Finally, in} Sec.~\ref{sec:conclusions} discussions and conclusions  \pro{are presented}. An appendix in Sec.~\ref{sec:cliff} has been included in order to complement the calculations through the article.
