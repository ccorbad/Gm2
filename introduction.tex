\section{Introduction}

The Standard Model (SM) of particle physics\out{,} is a very successful and predictive theory of fundamental particles and \out{forces} \pro{interaction} of nature. Recently ATLAS and CMS experiments discovered a new particle with approximate mass of \SI{125.6}{\GeV}, that is consistent with the SM Higgs boson~\cite{Aad:2012tfa,Chatrchyan:2012ufa}. This discovery \out{realize} \pro{will shed light on} the mechanism behind the Electroweak Symmetry Breaking (EWSB) and the nature of fermion masses\out{ through Yukawa couplings of fermion fields with the Higgs boson}. \pro{Although this} \out{Despite that this new particle} discovery \pro{represents} \out{is} an important \pro{success} \out{goal} from the theoretical point of view, there \pro{are} \out{still exists} problems \out{about} \pro{such as} the hierarchy of the fundamental interactions and \out{particle masses. This fact and} the lack of compatibility between Gravitation with SM. These have driven \out{to the High Energy Physics (HEP) community} to belive that the SM is an effective theory of a more fundamental one.

Many efforts has been made in the past century %In order to walk
to find a more general theory gravity. One of the extensions is the Cartan-Einstein Formalism (CEF), \out{(also known as first order formalism)} in which the independence between the connection and metric  is assumed\out{ at the beginning}. This theory \out{have no restriction about the symmetric aspect of the connection and} contains an antisymmetric part called \textit{torsion}. Including torsion in a pure gravity action, leads to the same results as in Einstein-Hilbert theory of Gravitation, but including fermionic matter into the picture, gives rise\out{s naturally} to a four fermion interaction from the equations of motion (see Ref.~\cite{Hehl:1976kj}). This effect \out{disagrees} \pro{contrasts} with the prediction of General Relativity (GR) coupled with fermions\pro{, therefore} \out{and} the experimental data will discern \out{if} \pro{whether or not} CEF is the correct description of Gravitation.

In four-dimensional spacetime, this four fermion interaction is highly suppressed by the inverse square of the \out{$4$-dimensional} Planck mass ($M_{\text{Pl}}\sim\si{10^{19}}{GeV}$). This \out{fact constraints} \pro{suppresses}  the effect of torsion \out{to be negligible} in four dimensions and complicated the measure and distinction between theories \out{of the correct formalism}. Extra dimensions models  offer an scenario without the hierarchy \pro{problem}~\cite{Randall:1999ee,ArkaniHamed:1998rs} \out{between Gravitational and SM interaction in the bulk}. \uline{The main difference between the couplings of the interactions and its hierarchy can be understood as a geometrical nature of the spacetime, relaxing the arbitrariness of the Yukawa's couplings in the SM.} \textsc{No se entiende!!!!!}

The aim of  this article is to \out{exploit} \pro{take advantage of} this four fermion interaction and explore one-loop observables using an effective theory coming from extra dimensions. These observables can be constrained using data of the precision tests of the SM~\cite{Altarelli:2004fq,Beringer:1900zz}, and limits over parameters of extra dimensions model can be achieved. 

The article is organized as follow: In Sec.~\ref{sec:CEF} \out{a brief introduction to} the Cartan-Einstein Formalism \pro{is introduced} and notation is \pro{fixed} \out{shown}. In Sec.~\ref{sec:extradim} the model \pro{is presented} and the effective theory in four-dimensions \pro{obtained} \out{coming from the extra dimensions scenario is performed}. In Sec.~\ref{sec:oneloop} the one-loop observables and form factor of this model are obtained. In Sec.~\ref{sec:constraints} limits coming from precision tests of the SM will be used to constraint the parameters of extra dimensions model. \pro{Finally, in} Sec.~\ref{sec:conclusions} discussions and conclusions \out{about the results obtained in the present work is done} \pro{are presented}. An appendix in Sec.~\ref{sec:cliff} has been included in order to complement the calculations through the article.
