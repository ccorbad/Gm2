\documentclass[twocolumn,showpacs,showkeys,prd,superscriptaddress]{revtex4-1}

\usepackage{float}
\usepackage{subfigure}
\usepackage{siunitx}
\usepackage{ragged2e}

%%%%%%%%% CORRECTIONS %%%%%%%%%
\usepackage{ulem}
\newcommand\pro[1]{{\color{blue}#1}}
\newcommand\out[1]{{\color{red}\sout{#1}}}
%%%%%%%%%%%%%%%%%%%%%%%%%%%%%%%

%---------Packages-------------
\usepackage{amsmath,amssymb,amsfonts,dsfont,mathrsfs,amsthm}
\usepackage{graphicx}
\usepackage{bbold}
\usepackage{slashed}
\usepackage{centernot}
\usepackage{hyperref}
\usepackage{lmodern}
\usepackage{xcolor}
\usepackage{comment}
\usepackage{epstopdf}
\hypersetup{linktocpage,colorlinks=true,urlcolor=blue,linkcolor=blue,citecolor=red}
\usepackage{feynmf}
\usepackage{array}

%---------Theorems------------
\newtheorem{Def}{Definition}
\newtheorem{Thm}{Theorem}
\newtheorem{Lem}{Lemma}
\newtheorem{Pos}{Postulate}
\newtheorem{Exa}{Example}
\newtheorem{Cor}{Corrolary}
\newtheorem{Pro}{Proposition}

%-------New Commands--------
\newcommand{\qd}{\textquestiondown}
\newcommand{\titulo}[1]{\Huge\textbf{#1}}
\newcommand{\lagrange}[1]{\frac{\partial\Lag}{\partial #1} - \frac{d}{dt}\frac{\partial\Lag}{\partial\dot{#1}} = 0}
\newcommand{\parcial}[2]{\frac{\partial #1}{\partial #2}}
\newcommand{\parciald}[2]{\frac{\delta #1}{\delta #2}}
\newcommand{\lagranged}[1]{\frac{\delta\Lag}{\delta #1} - \partial_\mu\frac{\delta\Lag}{\delta\left(\partial_\mu #1\right)} = 0}
\newcommand{\espacio}{\,\,\,\,\,}
\newcommand{\A}{\mathcal{A}} 
\newcommand{\abs}[1]{\left|#1\right|}
\newcommand{\C}{\mathbb{C}}
\newcommand{\bboxed}[1]{{\color{red}{\boxed{\boxed{\textcolor{black}{#1}}}}}}
\newcommand{\D}{\mathscr{D}}
\newcommand{\J}{\mathscr{J}}
\newcommand{\Lag}{\mathscr{L}}
\newcommand{\Lap}{\nabla^2}
\newcommand{\ket}[1]{\left.\left|#1\right.\right>}
\newcommand{\bra}[1]{\left.\left<#1\right.\right|}
\newcommand{\bkt}[3]{\left<#1\left|#2\right|#3\right>}
\newcommand{\bk}[2]{\left<#1\left|#2\right.\right>}
\newcommand{\comm}[2]{\left[#1,#2\right]}
\newcommand{\anticomm}[2]{\left\{#1,#2\right\}}
\newcommand{\vev}[1]{\ensuremath{\left<#1\right>}}
\newcommand{\uf}[2]{\ensuremath{u\(\vec{#1},#2\)}}
\newcommand{\ufb}[2]{\ensuremath{\bar{u}\(\vec{#1},#2\)}}
\newcommand{\vf}[2]{\ensuremath{v\(\vec{#1},#2\)}}
\newcommand{\vfb}[2]{\ensuremath{\bar{v}\(\vec{#1},#2\)}}
\newcommand{\ann}[3]{\ensuremath{#1\(\vec{#2},#3\)}}
\newcommand{\cre}[3]{\ensuremath{#1^\dag\(\vec{#2},#3\)}}
\newcommand{\vif}[1]{{\bf{e}}^{{#1}}}
\newcommand{\vifh}{\hat{{\bf{e}}}}
\newcommand{\vifhn}[2]{\hat{{\bf{e}}}^{\hat{#1}_{#2}}}
\newcommand{\etah}[2]{{\eta}_{\hat{#1}\hat{#2}}}
\newcommand{\etahn}[4]{{\eta}_{\hat{#1}_{#2}\hat{#3}_{#4}}}
\newcommand{\etahnu}[4]{{\eta}^{\hat{#1}_{#2}\hat{#3}_{#4}}}
\newcommand{\dduhn}[4]{\delta_{\hat{#1}_{#2}}^{\hat{#3}_{#4}}}
\newcommand{\vih}{\hat{e}}
\newcommand{\wfh}{\hat{{\boldsymbol{\omega}}}}
\newcommand{\wfhn}[4]{\hat{{\boldsymbol{\omega}}}^{\hat{#1}_{#2}\,\hat{#3}_{#4}}}
\newcommand{\wfudhn}[4]{\hat{{\boldsymbol{\omega}}}^{\hat{#1}_{#2}}{}_{\hat{#3}_{#4}}}
\newcommand{\wfhfree}{\hat{\mathring{{\boldsymbol{\omega}}}}}
\newcommand{\wfhnfree}[4]{\hat{\mathring{{\boldsymbol{\omega}}}}^{\hat{#1}_{#2}\hat{#3}_{#4}}}
\newcommand{\wfhnudfree}[4]{\hat{\mathring{{\boldsymbol{\omega}}}}^{\hat{#1}_{#2}}{}_{\hat{#3}_{#4}}}
\newcommand{\Rf}[2]{{\boldsymbol{\mathcal{R}}}^{#1 #2}}
\newcommand{\Rfhn}[4]{\hat{{\boldsymbol{\mathcal{R}}}}^{\hat{#1}_{#2}\hat{#3}_{#4}}}
\newcommand{\Rfhnfree}[4]{\hat{\mathring{{\boldsymbol{\mathcal{R}}}}}^{\hat{#1}_{#2}\hat{#3}_{#4}}}
\newcommand{\Tf}[1]{{\boldsymbol{\mathcal{T}}}^{#1}}
\newcommand{\Th}{\hat{\mathcal{T}}}
\newcommand{\Tfhn}[2]{\hat{{\boldsymbol{\mathcal{T}}}}^{\hat{#1}_{#2}}}
\newcommand{\hodge}{\star}
\newcommand{\K}{\mathcal{K}}
\newcommand{\Ma}{\mathcal{M}}
\newcommand{\R}{\mathcal{R}}
\newcommand{\Rh}{\hat{{\mathcal{R}}}}
\newcommand{\etahd}[2]{\hat{\eta}_{\hat{#1}\hat{#2}}}
\newcommand{\etahu}[2]{\hat{\eta}^{\hat{#1}\hat{#2}}}
\newcommand{\Rfh}{\hat{{\boldsymbol{\mathcal{R}}}}}
\newcommand{\Kf}{{\boldsymbol{\mathcal{K}}}}
\newcommand{\Kfhn}[4]{\hat{{\boldsymbol{\mathcal{K}}}}^{\hat{#1}_{#2}\hat{#3}_{#4}}}
\newcommand{\Kfhnud}[4]{\hat{{\boldsymbol{\mathcal{K}}}}^{\hat{#1}_{#2}}{}_{\hat{#3}_{#4}}}
\newcommand{\Kfuu}[2]{{\boldsymbol{\mathcal{K}}}^{#1 #2}}
\newcommand{\Kfdd}[2]{{\boldsymbol{\mathcal{K}}}_{#1 #2}}
\newcommand{\Kfud}[2]{{\boldsymbol{\mathcal{K}}}^{#1}{}_{#2}}
\newcommand{\wf}{{\boldsymbol{\omega}}}
\newcommand{\wfb}{\bar{{\boldsymbol{\omega}}}}
\newcommand{\Df}{{\boldsymbol{D}}}
\newcommand{\df}{{\boldsymbol{d}}}
\newcommand{\gf}{{\boldsymbol{\gamma}}}
\newcommand{\Af}{{\boldsymbol{A}}}
\newcommand{\T}{\mathcal{T}}
\newcommand{\free}[1]{\mathring{#1}}
\newcommand{\ghu}[1]{\gamma^{\hat{#1}}}
\newcommand{\ghhhu}[3]{\gamma^{\hat{#1}\hat{#2}\hat{#3}}}
\newcommand{\ghhhd}[3]{\gamma_{\hat{#1}\hat{#2}\hat{#3}}}
\newcommand{\1}{\mathbb{1}}
\newcommand{\gs}{\gamma^{*}}
\newcommand{\form}[1]{{\boldsymbol{#1}}}

%--------------Operators------------
\newcommand{\diag}{\operatorname{diag}}
\newcommand{\tr}{\operatorname{tr}}
\newcommand{\Tr}{\operatorname{Tr}}
\newcommand{\Ker}{\operatorname{Ker}}
\renewcommand{\Im}{\operatorname{Im}}
\newcommand{\sgn}{\operatorname{sgn}}
\newcommand{\Ln}{\operatorname{Ln}}
\newcommand{\Ei}{\operatorname{Ei}}
\newcommand{\csch}{\operatorname{csch}}
\newcommand{\arcsinh}{\operatorname{arcsinh}}

\newcommand\dv[1][]{\ensuremath{\mathrm{d}V_{\! #1}}}

\usepackage{tikz}
\usetikzlibrary{arrows,shapes,positioning}
\usetikzlibrary{decorations.markings}
\usetikzlibrary{decorations.pathreplacing}
\usetikzlibrary{decorations.pathmorphing}
\tikzstyle directed=[postaction={decorate,decoration={markings,mark=at position .5 with {\arrow{stealth}}}}]
\tikzset{%
  cross/.style={path picture={ 
      \draw[black]
      (path picture bounding box.south east) -- (path picture bounding box.north west) 
      (path picture bounding box.south west) -- (path picture bounding box.north east);
}}}
\tikzset{boson/.style={decorate,decoration={snake}}}



\begin{document}
%\title{Torsional Contribution to One-Loop Observables in an Effective Theory Coming From Extra Dimensions}
\title{Torsion in Extra Dimensions and One-Loop Observables}

\author{Crist\'obal \surname{Corral}}
\email{cristobal.corral@postgrado.usm.cl}
\affiliation{Departamento de F\'\i sica, Universidad T\'ecnica Federico Santa Mar\'\i a, Casilla 110-V, Valpara\'\i so, Chile.}

\author{Oscar \surname{Castillo-Felisola}}
\email{o.castillo.felisola@gmail.com} 
\affiliation{Departamento de F\'\i sica, Universidad T\'ecnica Federico Santa Mar\'\i a, Casilla 110-V, Valpara\'\i so, Chile.}
\affiliation{Centro Cient\'ifico Tecnol\'ogico de Valpara\'iso, Chile.}

\author{Sergey \surname{Kovalenko}}
\email{sergey.kovalenko@usm.cl}  
\affiliation{Departamento de F\'\i sica, Universidad T\'ecnica Federico Santa Mar\'\i a, Casilla 110-V, Valpara\'\i so, Chile.}
\affiliation{Centro Cient\'ifico Tecnol\'ogico de Valpara\'iso, Chile.}

\author{Iv\'an \surname{Schmidt}}
\email{ivan.schmidt@usm.cl}  
\affiliation{Departamento de F\'\i sica, Universidad T\'ecnica Federico Santa Mar\'\i a, Casilla 110-V, Valpara\'\i so, Chile.}
\affiliation{Centro Cient\'ifico Tecnol\'ogico de Valpara\'iso, Chile.}

\begin{abstract}
  \out{In the present work, Cartan-Einstein formulation of gravity in extra dimension scenario have been considered. Non zero torsion have been included by considering gravity coupled with fermions in the minimal extension of this theory. This consideration gives origin to a four fermion interaction in a natural way.}
  The aim of this article is to explore one-loop observables due the four fermion interaction in a effective theory coming from the torsionful extra dimension scenario. The results are compared with Standard Model precision tests, allowing  to constraint the scale for new physics and the parameters of the model.
\end{abstract}

\maketitle


\section{Introduction}

The Standard Model (SM) of particle physics is a very successful and predictive theory of fundamental particles and  \pro{interaction} of nature. Recently ATLAS and CMS experiments discovered a new particle with approximate mass of \SI{125.6}{\GeV}, that is consistent with the SM Higgs boson~\cite{Aad:2012tfa,Chatrchyan:2012ufa}. This discovery  \pro{will shed light on} the mechanism behind the Electroweak Symmetry Breaking (EWSB) and the nature of fermion masses. \pro{Although this}  discovery \pro{represents}  an important \pro{success}  from the theoretical point of view, there \pro{are}  problems  \pro{such as} the hierarchy of the fundamental interactions and  the lack of compatibility between Gravitation with SM. These have driven  to belive that the SM is an effective theory of a more fundamental one.

Many efforts have been made in the past century %In order to walk
to find a more general theory gravity. One of the extensions is the Cartan-Einstein Formalism (CEF),  in which the independence between the connection and the metric  is assumed. This theory  contains an antisymmetric part of the connection called \textit{torsion}. Including torsion in a pure gravity action, leads to the same results as in Einstein-Hilbert theory of Gravitation, but including fermionic matter into the picture, gives rise to a four fermion interaction  (see Ref.~\cite{Hehl:1976kj}). This effect  \pro{contrasts} with the prediction of General Relativity (GR) coupled with fermions\pro{, therefore}  the experimental data will discern  \pro{whether} CEF is the correct description of Gravitation.

In four-dimensional spacetime, this four fermion interaction is highly suppressed by the inverse square of the  Planck mass ($M_{\text{Pl}}\sim\si{10^{19}}{GeV}$). This  \pro{suppresses}  the effect of torsion  in four dimensions, and hinders the  distinction between theories . Extra dimensional models  offer an scenario without the hierarchy \pro{problem}~\cite{Randall:1999ee,ArkaniHamed:1998rs}. \uline{The main difference between the couplings of the interactions and its hierarchy can be understood as a geometrical nature of the spacetime, relaxing the arbitrariness of the Yukawa's couplings in the SM.} \textsc{No se entiende!!!!!}

The aim of  this article is to  \pro{take advantage of} this four fermion interaction and explore one-loop observables using an effective theory coming from extra dimensions. These observables can be constrained using data of the precision tests of the SM~\cite{Altarelli:2004fq,Beringer:1900zz}, and limits over parameters of extra dimensions model can be achieved. 

The article is organized as follow: In Sec.~\ref{sec:CEF}  the Cartan-Einstein Formalism \pro{is introduced} and notation is \pro{fixed} . In Sec.~\ref{sec:extradim} the model \pro{is presented} and the effective theory in four-dimensions \pro{obtained/performed} . In Sec.~\ref{sec:oneloop} the one-loop observables and form factor of this model are obtained. In Sec.~\ref{sec:constraints} limits coming from precision tests of the SM will be used to constraint the parameters of extra dimensions model. \pro{Finally, in} Sec.~\ref{sec:conclusions} discussions and conclusions  \pro{are presented}. An appendix in Sec.~\ref{sec:cliff} has been included in order to complement the calculations through the article.

\section{Cartan-Einstein theory of gravity with fermions}
In this section a brief description of Cartan-Einstein Formulation (CEF) of gravity will be done. CEF is constructed in more general way that Einstein-Hilbert one. Spin connection and vielbeins are independent at the beginning and the presence of torsion is also assumed. Including torsion into the picture implies that one will assume nonvanishing antisymmetric part for the affine connection 
\begin{align}
 \hat{T}_{\hat{a}}{}^{\hat{b}}{}_{\hat{c}} \equiv \hat{\Gamma}_{[\hat{a}}{}^{\hat{b}}{}_{\hat{c}]} = \hat{\Gamma}_{\hat{a}}{}^{\hat{b}}{}_{\hat{c}} - \hat{\Gamma}_{\hat{c}}{}^{\hat{b}}{}_{\hat{a}}
\end{align}
where latin indices with a hat denotes local inertial frame coordinates, running over $D$-dimensions. This local inertial frame is described mathematically by (co)tangent space ($T_P^*M$) $T_PM$ at each point $P$ on the manifold. Tangent space coordinates can be mapped into full $D$-dimensional spacetime coordinates $\hat{\mu}$, endowed with some curved metric $\hat{g}_{\hat{\mu}\hat{\nu}}(x)$, using vielbeins fields and the relation
\begin{equation}
\label{metricrelation}
 \hat{g}_{\hat{\mu}\hat{\nu}}(x) = \hat{\eta}_{\hat{a}\hat{b}}\,\vih^{\hat{a}}_{\hat{\mu}}(x)\,\vih^{\hat{b}}_{\hat{\nu}}(x).
 \end{equation}
where $\etah{a}{b} = \diag{\left(-,+,+,+,\ldots,+\right)}$ is the Minkowski metric in $D$-dimensional spacetime and $\vih(x)^{\hat{a}}_{\hat{\mu}}$ are the vielbein fields. If one wants to add fermionic matter and preserve the local Lorentz invariance of
\begin{equation}
 S_\Psi = \int d^Dx\,\Lag\left(\Psi,\partial_{\hat{\mu}}\Psi\right),
\end{equation}
one defines the covariant derivative for fermions and the covariant volume element. This is achieved by replacing
\begin{equation}
  S_\Psi = \int \dv[D]%d^Dx\,|\hat{e}|
  \,\Lag\left(\Psi,D_{\hat{a}}\Psi\right),
\end{equation}
where \dv[D] %$$=d^Dx\,|\hat{e}| = d^Dx\,\sqrt{|\hat{g}|},$$ 
is the covariant volume element defined by
\begin{align}
  \dv[D]=d^D\! x\,|\hat{e}|= d^D\! x\,\sqrt{|\hat{g}|},
\end{align}
with $|\hat{e}|$ %=\det{\left(\vih^{\hat{a}}_{\hat{\mu}}\right)}$ 
the determinant of the $D$-dimensional vielbein and
\begin{equation}
\label{covder}
 D_{\hat{a}}\Psi = \hat{E}_{\hat{a}}^{\hat{\mu}}D_{\hat{\mu}}\Psi = \hat{E}_{\hat{a}}^{\hat{\mu}}\left(\partial_{\hat{\mu}}\Psi + \frac{1}{4}(\omega_{\hat{\mu}})^{\hat{b}\hat{c}}\gamma_{\hat{b}\hat{c}}\Psi\right)
\end{equation}
denotes the covariant derivative for fermions. In Eq.~\eqref{covder} $(\omega_{\hat{\mu}})^{\hat{b}\hat{c}}$ denotes the spin connection twisted by presence of torsion, $\gamma_{\hat{a}\hat{b}} = \frac{1}{2}\comm{\gamma_{\hat{a}}}{\gamma_{\hat{b}}}$ and $\hat{E}^\mu_a = \left(\hat{e}^a_\mu\right)^{-1}$ is the inverse of the $D$-dimensional vielbein. Using this version of fermionic matter action, the local Lorentz invariance is guaranteed. The torsionful spin connection can be decomposed into
\begin{equation}
\label{spinconndecomp}
\wfh^{\hat{a}\hat{b}} = \wfhfree^{\hat{a}\hat{b}} + \hat{\Kf}^{\hat{a}\hat{b}},
\end{equation}
where $\wfhfree^{\hat{a}\hat{b}} = (\hat{\mathring{\omega}}_{\hat{\mu}})^{\hat{a}\hat{b}}\,dx^{\hat{\mu}}$ denotes the one-form torsion-free spin connection and $\hat{\Kf}^{\hat{a}\hat{b}} = \hat{\K}_{\mu}{}^{\hat{a}\hat{b}}\,dx^{\hat{\mu}}$ is the {\it{contorsion}} one-form and contains torsion information. Hereon circled quantities are torsion-free and bold symbols denote differential forms.  Contorsion and torsion tensors are related via
\begin{equation}
\label{generalcontorsion}
\hat{\K}_{\hat{a}}{}^{\hat{b}}{}_{\hat{c}} = \frac{1}{2}\left(\hat{\T}_{\hat{a}}{}^{\hat{b}}{}_{\hat{c}} - \hat{\T}_{\hat{a}\hat{c}}{}^{\hat{b}} + \hat{\T}^{\hat{b}}{}_{\hat{a}\hat{c}}\right).
\end{equation}

The two most important equations in this formalism, that relates the vielbeins and the spin connection with torsion and curvature respectively are the so called {\it{Cartan structure equations}} 
\begin{align}
\label{cartantorsion}
 \df\vifhn{a}{} + \wfh^{\hat{a}}{}_{\hat{c}}\wedge\vifhn{c}{} &= \Tfhn{a}{}, \\
 \label{cartancurvature}
 \df\wfh^{\hat{a}\hat{b}} + \wfh^{\hat{a}}{}_{\hat{c}}\wedge\wfh^{\hat{c}\hat{b}} &= \Rfhn{a}{}{b}{},
\end{align}
where $\vifhn{a}{} \equiv \vih^{\hat{a}}_{\hat{\mu}}\,dx^{\hat{\mu}}$ is the vielbein one-form, $\Tfhn{a}{}$ and $\Rfhn{a}{}{b}{}$ are the torsion and curvature two-forms respectively, defined by
\begin{align}
 \Tfhn{a}{} &= \frac{1}{2!}\hat{\T}_{\hat{\mu}}{}^{\hat{a}}{}_{\hat{\nu}}\,dx^{\hat{\mu}}\wedge dx^{\hat{\nu}},\\
 \Rfhn{a}{}{b}{} &= \frac{1}{2!}\Rh^{\hat{a}\hat{b}}{}_{\hat{\mu}\hat{\nu}}\,dx^{\hat{\mu}}\wedge dx^{\hat{\nu}}.
\end{align}


\subsection{Action and equations of motion}
The following action will be considered
\begin{align}
\nonumber
 S &= \frac{1}{2\kappa_*^2}\int\frac{\epsilon_{\hat{a}_1\ldots\hat{a}_D}}{(D-2)!}\,\Rfh^{\hat{a}_1\hat{a}_2}\wedge\vifh^{\hat{a}_3}\wedge\ldots\wedge\vifh^{\hat{a}_D} \\
   \nonumber
   &\quad+ \sum_{r}\int\bigg(\frac{1}{2}\left((-1)^D\bar{\Psi}_r\hodge\gf\wedge\Df\Psi_r + \Df\bar{\Psi}_r\wedge\hodge\gf\Psi_r\right) \\
 \label{formaction}
 &\qquad- m\bar{\Psi}_r\Psi_r\,\frac{\epsilon_{\hat{a}_1\ldots\hat{a}_D}}{D!}\vifh^{\hat{a}_1}\wedge\ldots\wedge\vifh^{\hat{a}_D}\bigg)
\end{align}
where $\kappa_*^2 = 8\pi G_* = M_*^{-(2+n)}$ ($G_*$ and $M_*$ are the Newtonian gravity constant and the reduced Planck mass on $D = 4 + n$ dimensions respectively). In the fermionic sector $\Psi_r$ denotes the spinor field of flavor $r$ on the $D$-dimensional spacetime with $2^{[D/2]}$ components. The definitions $\gf = \gamma_{\hat{a}_1}\vifh^{\hat{a}_1}$ and $\Df\Psi = (D_{\hat{b}_1}\Psi)\,\vifh^{\hat{b}_1}$ has been used and $\hodge\gf$ denotes {\it{Hodge Dual}} of $\gf$. The adventage of using differential forms, is that this formalism is coordinate independent and the equations of motion are easily found. The action in Eq.~\eqref{formaction} is equivalent to
\begin{align}
\nonumber
 S &= \frac{1}{2\kappa_*^2}\int \dv[D]\,\hat{\R} \\ 
 \nonumber
 &\quad -\sum_{r}\int \dv[D]\,\bigg(\frac{1}{2}\left(\bar{\Psi}_r\gamma^{\hat{a}}D_{\hat{a}}\Psi_r - D_{\hat{a}}\bar{\Psi}\gamma^{\hat{a}}\Psi\right)\\
  \label{coordinateaction}
 &\qquad + m\bar{\Psi}_r\Psi_r\bigg).
\end{align}

The equations of motion are found using the principle of least action. Varying with respect to the vielbein leads to the Einstein equations
\begin{equation}
 \label{einsteineom}
 \Rh_{\hat{a}\hat{b}} - \frac{1}{2}\etah{a}{b}\Rh = \kappa_*^2\,\hat{T}_{\hat{a}\hat{b}}
\end{equation}
where $\hat{T}_{\hat{a}\hat{b}}$ denotes the energy-momentum tensor of fermions. For this fermionic action the energy-momentum tensor is
\begin{align}
 \hat{T}_{\hat{a}\hat{b}} &= \sum_{r}\frac{1}{2}\left(\bar{\Psi}_r\gamma_{\hat{a}}D_{\hat{b}}\Psi_r - D_{\hat{b}}\bar{\Psi}_r\,\gamma_{\hat{a}}\Psi_r\right) - \etah{a}{b}\Lag_{\Psi}
\end{align}
with $\Lag_\Psi$ the Dirac Lagrangian from Eq.~\eqref{coordinateaction}. The second equation of motion can be obtained varying with respect to the torsionful spin connection, giving
\begin{align}
\label{torsion}
 \Th_{\hat{a}}{}^{\hat{b}}{}_{\hat{c}} = -\, \frac{\kappa_*^2}{2}\sum_{r}\bar{\Psi}_r\gamma_{\hat{a}}{}^{\hat{b}}{}_{\hat{c}}\Psi_r.
\end{align}
%% One could notice that gravity and fermions in CEF leads to a completely antisymmetric torsion tensor. 
Using Eq.~\eqref{generalcontorsion} the contorsion tensor is  found to be
\begin{align}
\label{contorsionfound}
\hat{\K}_{\hat{a}}{}^{\hat{b}}{}_{\hat{c}} &= + \frac{\kappa_*^2}{4}\sum_{r}\bar{\Psi}_r\gamma_{\hat{a}}{}^{\hat{b}}{}_{\hat{c}}\Psi_r.
\end{align}

\subsection{Torsional contribution to the fermionic action}

%% Eq.~\eqref{contorsionfound} has no dynamics, therefore one can sustitute into the initial action as a constraint. There exists models where torsion appears as a dynamical field instead of the minimal consideration of this work (for further reading Ref.~\cite{Carroll:1994dq,Belyaev:1998ax}). Replacing Eq.~\eqref{contorsionfound} in gravity sector leads to
Equation of motion of the spin connection is a constraint, therefore it can be substituted into the initial action.
%% There exists models where torsion appears as a dynamical field instead of the minimal consideration of this work (for further reading Refs.~\cite{Carroll:1994dq,Belyaev:1998ax}). 
Replacing Eq.~\eqref{contorsionfound} in gravity sector leads to  
\begin{align}
 \label{gravdecomp}
 \Rfhn{a}{1}{a}{2} = \Rfhnfree{a}{1}{a}{2} + \free{\Df}\Kfhn{a}{1}{a}{2} + \Kfhnud{a}{1}{b}{}\wedge\Kfhn{b}{}{a}{2}.
\end{align}
For Dirac fermions, this replacement adds a contribution in the covariant derivative 
\begin{align}
 \Df\Psi_r = \free{\Df}\Psi_r + \frac{1}{4}\Kfhn{a}{}{b}{}\,\gamma_{\hat{a}\hat{b}}\Psi_r.
\end{align}
Considering these decompositions in the initial action, one finds
\begin{align}
\label{4FI}
 S = \free{S}_\text{grav} + \free{S}_\Psi + \frac{\kappa_* ^2}{32}\sum_{r,s}\int \dv[D]\,\bar{\Psi}_r\gamma^{\hat{a}\hat{b}\hat{c}}\Psi_r\,\bar{\Psi}_s\gamma_{\hat{a}\hat{b}\hat{c}}\Psi_s
\end{align}


From Eq.~\eqref{4FI}, one can notice that a more general formulation of gravity, with nonvanishing torsion tensor is considered, leads to a four fermion interaction. Several authors also have studying possible gravitational effects of torsion and its possible phenomenology Ref.~\cite{Belyaev:1998ax,Fabbri:2010hz,Capozziello:2012xt,Mavromatos:2012cc,CastilloFelisola:2012fy,Fabbri:2013gza,Kostelecky:2007kx}.
%A more general formulation of gravity, where nonvanishing torsion tensor is considered, leads to a four fermionic interaction as one can notice in Eq.~\eqref{4FI}. Several authors also have studying possible gravitational effects of torsion and its possible phenomenology Ref.~\cite{Belyaev:1998ax,Fabbri:2010hz,Capozziello:2012xt,Mavromatos:2012cc,CastilloFelisola:2012fy,Fabbri:2013gza}.

Two features that should be highlighted. First, the universality of this four fermion interaction due to flavor blindness of gravity. Second, torsion arises from the kinetic term in the Dirac action. This fact constraints the quantum numbers per current and fermions in this interaction appears in pairs. 

%ambiguity of theory with torsion..

It is important to remark that, instead of the ambiguity of choice for theory with torsion, the minimal extension of CEF has been used. In this framework gauge fields does not couple to torsion at classical level. If one consider gauge fields on curved spacetimes, the field strength tensor appears as torsion-free term in the action, and there is no reason to change partial derivatives by torsionful covariant derivatives in the field strength tensor for gauge bosons, because the usual definition
\begin{align}
\hat{F}_{\hat{\mu}\hat{\nu}}^a &= \partial_{\hat{\mu}}\hat{A}_{\hat{\mu}}^a - \partial_{\hat{\nu}}\hat{A}_{\hat{\nu}}^a +f^{a}{}_{bc}\hat{A}^b_{\hat{\mu}}\hat{A}^c_{\hat{\nu}}
\end{align}
is also valid for curved spacetimes. A more natural way to understand this, relies on CEF, because one-form gauge field transform as a connection (in CEF formalism, gauge fields are also one-forms). By this reason active Lorentz transformation keeps invariant the one-form gauge fields, and no covariant derivative is needed (for a further reading, see Ref.~\cite{Benn:1980ea}). This condition keeps safe the gauge invariance of the theory Ref.~\cite[p.407]{Hehl:1976kj}. One could do explicit calculation with torsionful covariant derivative acting on one-form gauge field, and realize that the gauge violating terms vanishes.




\section{Extra dimensions scenario}
\subsection{Motivations}

In the previous section, a four fermionic interaction due torsion has been found. One of the problems of this interaction is that, if one consider $4$-dimensional scenario, the coupling constant $\kappa^2$ is suppressed by $4$-dimensional Planck mass. There exist extra dimensions models where the $4$-dimensional Planck scale $M_{\text{Pl}}$ is an effective one from a fundamental Planck scale $M_*$ that relies in a extra dimensions manifold. Some of these models has been proposed by Arkani-Hamed, Dimopoulos and Dvali (ADD) Ref.~\cite{ArkaniHamed:1998rs}, that consider $n\geq2$ compact extra dimensions and by Randall and Sundrum (RS) Ref.~\cite{Randall:1999ee} that assumes $n=1$ large extra dimensions. 

In extra dimensions manifolds, one has to decompose the higher dimensional spinor $\Psi$ into a four dimensional ones 
 \begin{align}
 \label{spinordecomp}
  \Psi(x,\xi) &= N\,\sum_{i}\; \psi^{(i)}(x)\chi_{(i)}(\xi)
 \end{align}
where $\psi(x)$ and $\chi(\xi)$ are the four and $n$-dimensional spinors respectively ($x$ and $\xi$ denotes the four and $n$-dimensional coordinates respectively) and $N$ is a normalization factor. In order to get a effective theory in $4$-dimensions from a fundamental $D = 4 + n$ dimensions theory, one has to perform dimensional reduction of the $n$ extra dimensions.

\subsection{The model}
In the present work, RS metric Ref.~\cite{Randall:1999ee} will be consider
\begin{align}
 \label{RSmetric}
 ds^2 = e^{-2ky}\eta_{\mu\nu}dx^\mu\,dx^\nu + dy^2.
\end{align}

In this scenario, the fifth dimension $y$ is compactified on an orbifold, $S^1/\mathbb{Z}_2$ of radius $R$ in the interval $0\leq y\leq \pi R$. The SM fields are localized on IR brane (this set up, is similar to Ref.~\cite{Gherghetta:2000qt,Gherghetta:2006ha} but with four fermion interaction coming from torsionful manifold). One can identify the $5$-dimensional vielbein directly from Eq.~\eqref{RSmetric} 
\begin{align}
\vifhn{a}{} &\equiv \left(\vifh^{a},\vifh^5\right) = \left(e^{-k|y|}\,dx^\mu,dy\right), 
\end{align}
and the invariant volume element can be calculated using 
\begin{align}
 |\hat{e}| = \det{\vih(x)^{\hat{a}}_{\hat{\mu}}} = e^{-4k|y|}.
\end{align}
Note that the determinant of the vielbein has only dependence on the fifth dimension and can be integrated out when dimensional reduction is performed. 

The Kaluza-Klein (KK) decomposition for gauge and fermionic fields respectively are
\begin{align}
\label{KKgaugedecomp}
\hat{A}_{\mu}^a(x,y) &= \frac{1}{\sqrt{\pi R}}\sum_{i}h^{(i)}(y)A_{\mu\,(i)}^a(x), \\
\label{KKspindecomp}
 \Psi(x,y)_r &= \frac{1}{\sqrt{\pi R}}\sum_{i}f_r^{(i)}(y)\psi_r^{(i)}(x),
\end{align}
where $R$ denotes the typical radius of the extra dimension, and the initial term is introduce as a normalization factor. The extra dimension information of $\Psi_r$ and $\hat{A}_\mu^a$ are enconded by $h^{(i)}(y)$ and $f_r^{(i)}(y)$ profiles. The terms $\psi_r^{(i)}(x)$ and $A_{\mu\,(i)}^a(x)$ denotes fermion of $r$ flavor and gauge fields on $4D$ spacetime, expanded on KK modes. 

In the previous KK decomposition for gauge fields, $A_4=0$ has been used. This choice eliminates $A_4$ from the 3-brane but the gauge invariance of the effective action in 4-dimension still remains (see Ref.~\cite{Davoudiasl:1999tf}). 

In the following analysis, only the zero mode of KK gauge and fermionic excitations $h^{(0)}(y)$ and $f_r^{(0)}(y)$ respectively, will be consider. This approach gives only the lower mass of KK tower and allows to search in the threshold of finding extra dimensions, because the upper modes with higher masses will be less accessible.

\subsection{Effective theory in $4$D}

Clifford algebra in $D = 5$ dimensions can be constructed using the $4$-dimensional one plus the chiral gamma matrix defined by $\gamma^5 = i\,\gamma^0\gamma^1\gamma^2\gamma^3$. Using tangent space coordinates, gamma matrices in $5D$ spacetime reads
\begin{align}
  \ghu{a} = \left(\gamma^a,\gamma^*\right).
\end{align}
With this definition the product of gamma matrices in Eq.~\eqref{4FI}, using a general $5$-dimensional spacetime, gives
\begin{align}
 \ghhhu{a}{b}{c}\ghhhd{a}{b}{c} &= \gamma^{abc}\gamma_{abc} + 3\gamma^{ab*}\gamma_{ab*} \\
 &= 6\left(\gamma_a\gamma^*\right)\left(\gamma^a\gamma^*\right) + 3\left(\gamma^{ab}\gamma^*\right)\left(\gamma_{ab}\gamma^*\right)
\end{align}
(a general treatment of this calculation has been done in Apendix I). 

If one consider only neutral gauge-fermion coupling in the $5D$ bulk as in Ref.~\cite{Davoudiasl:1999tf}, the interaction reads
\begin{align}
S_{\text{int}} &= -i\sum_{r}\,g_5\int d^5x\,|\hat{e}|\,\bar{\Psi}_r \gamma^{\hat{\mu}}T^a\Psi_r\hat{A}_\mu^a
\end{align}
where $T^a$ are the generators of Lie algebra associated to the gauge group. Defining the zero KK modes $f_r(y)\equiv f_r^{(0)}(y)$, $h^{(0)}(y)\equiv h(y)$, $\psi_r\equiv\psi_r^{(0)}(x)$ and $A_\mu\equiv A_{\mu\,(0)}(x)$ and considering only this contribution, the previous interaction reads
\begin{align}
S_{\text{int}} \approx -i \sum_{r}\,g_{\text{eff}}\,\int d^4x\,\bar{\psi}_r\gamma^\mu T^a\psi_r A_\mu^a
\end{align}
where the effective coupling
\begin{align}
 g_{\text{eff}} \equiv \int_0^{\pi R}dy\,e^{-4ky}g_5\,f_r^*(y)\,f_r(y)\,h(y)
\end{align}
contains all the information of the gauge coupling in the bulk.

The four fermionic interaction in Eq.~\eqref{4FI} coming from torsion, can be written as
\begin{align}
\nonumber
 S_{4\text{FI}} &= \frac{\kappa_{\text{eff}}^2}{32}\sum_{r,s}\int d^4x\left(6\,\bar{\psi}_{r}\gamma^\mu\gamma^5\psi_{r}\,\bar{\psi}_{s}\gamma_\mu\gamma^5\psi_{s}\right.\\
 \label{4FI5D}
 &+\left. 3\,\bar{\psi}_r\gamma^{\mu\nu}\gamma^5\psi_r\,\bar{\psi}_s\gamma_{\mu\nu}\gamma^5\psi_s\right)
\end{align}
where $\kappa_{\text{eff}}^2$ has been defined in terms of the zero modes of KK excitations, from Ref.~\cite{Gherghetta:2000qt}
\begin{align}
 k_{\text{eff}}^2 \simeq \frac{k}{M_*^3}\,e^{(4-2c_m-2c_n)\pi kR}.
\end{align}
and contains all the extra dimensions information after performing dimensional reduction. A special choice of the profiles can be done from the values obtained on Ref.~\cite{Gherghetta:2006ha}, in order to explore only hierarchy of the gravitation scale $M_*$ on extra dimensions. The choice $c_i\simeq1$ will be used, in order to deal only with the hierarchy of the gravitational scale. With this, one obtain
\begin{align}
 M_{\text{Pl}} \simeq \frac{M_*^3}{k}
\end{align}

For the same reason, the effective coupling of neutral gauge bosons coupled to fermionic current, will be considered equal as the SM in $4D$, (i.e.: $g_{\text{eff}} = e\,Q_f$ for photon exchange and $g_{\text{eff}} = e/(2c_Ws_W)\left(T^3_f - 2\,s_W^2\,Q_f\right)$ for $Z^0$ exchange, where $Q_f$ is electric charge in proton unities, $T_f^3$ is the third component of weak isospin and $s_W(c_W) = \sin(\cos)\theta_W$). 

An extra axial-tensor term arises by considering only one extra dimension in Eq.~\eqref{4FI5D}. This term will play an important role in the following section, because contribute to magnetic form factor and some experiments of anomalous magnetic moment, will be very sensitive to this contribution. (cites of electron magnetic moment, neutrino magnetic moment and muon magnetic moment).



\section{One loop calculations and form factors}

In this section, the effects of curvature in the effective theory in $4$-dimensions will be ignored, by the fact that the Universe is essentially flat as Ref.~\cite{Larson:2010gs} indicates. This assumption is also based in our comparision of torsion effects with particle accelerators data (the predominant forces in these experiments becames from the SM interactions, and the curvature effects are negligible). Obviously, this assumption is not valid anymore where curvature effects can not be droped, i.e.: near to a black hole or neutron star. This consideration has been used before in Ref.~\cite{Carroll:1994dq,Belyaev:1998ax}. In CEF the metric and connection are independent, this condition allows to curvature and torsion be independent too. There exist manifolds with torsion and no curvature, where teleparallel gravity relies (for further reading, see Ref.~\cite{Arcos:2005ec}). This special kind of manifolds are called Weitzenb\"ock manifolds.

Our interest by now, is try to extract information of the contribution of torsion to one loop form factors. Now, considering that $SU(2)_L\otimes U(1)_Y$ gauge sector of the SM is torsion free, the only effect of torsion is through the four fermion contact term in Eq.~\eqref{4FI5D}. Using this kind of interaction, our aim is to do one loop calculation in this theory. 

The process to be calculated in this section can be extracted from the general four femionic interaction Lagrangian used in Ref.~\cite{GonzalezGarcia:1998ay} plus one neutral gauge boson exchange. The relevant Lagrangians used by Gonzalez-Garc\'ia, Gusso and Novaes in the previous reference are
 \begin{align}
  \nonumber
   \Lag_{\text{V}} &= \eta_V\,\frac{g^2}{\Lambda^2}\left[\psi_r\gamma_\mu\left(V_V - A_V\gamma_5\right)\psi_r\right] \\ 
  \label{lagvec}
  &\espacio\espacio\espacio\times \left[\psi_s\gamma^\mu\left(V_V - A_V\gamma_5\right)\psi_s\right], \\
  \nonumber
    \Lag_{\text{T}} &= \eta_T\,\frac{g^2}{\Lambda^2}\left[\psi_r\sigma_{\mu\nu}\left(V_T - A_T\gamma_5\right)\psi_r\right]\\ 
     \label{lagten}
     &\espacio\espacio\espacio\times\left[\psi_s\sigma^{\mu\nu}\left(V_T - A_T\gamma_5\right)\psi_s\right],
 \end{align}
where $r$ and $s$ denotes flavor indices as in the previous sections and in the following, the notation of Ref.~\cite{GonzalezGarcia:1998ay} for evaluating form factors will be used. The most general one loop calculation in this theory, can be build with the previous four fermion interaction Lagrangian and gauge boson coupled to fermions
\begin{align}
\label{feyndiagram}
  \begin{tikzpicture}[thick,baseline=(current  bounding  box.center)]
    \coordinate (V) at (0,0);
    \node[circle,draw=black,shade,minimum size=.6cm]  at (V)  {};
    \draw[boson] (-2,0) node[anchor=south] {$V_\mu(k)$} -- (180:3mm);
    \draw[directed] (1,-1) node[anchor=west] {$f(p)$}  -- (-45:3mm);
    \draw[directed] (45:3mm) -- (1,1) node[anchor=west] {$f(p')$};
  \end{tikzpicture}
  =\imath e \, V_\mu(k) J^\mu(p,p')
\end{align}
%% \begin{figure}[H]
%% \scalebox{1.5}[1.5]{\begin{tikzpicture}
%%  \draw[boson,thick] (-5,0) -- (-3.3,0);
%%  \node[above,scale=.7] at (-4,.2) {$Z^0_\mu(k),A_\mu(k)$};
%%  \draw[thick,directed] (-2,-1) -- (-3,0);
%%  \node[right,scale=.7] at (-2.3,-.5) {$f(p)$};
%%  \draw[thick,directed] (-3,0) -- (-2,1);
%%  \node[right,scale=.7] at (-2.3,.5) {$f(p')$};
%%  \filldraw[shade] (-3,0) circle (.3);
%%  \node[scale=.6] at (-2,0) {$=$};
%%  \node[scale=.6] at (-.8,0) {$(i\,e)V_\mu(k) J^\mu(p,p')$};
%% \end{tikzpicture}}
%% \caption{Feynman diagram for one loop process including gauge coupling, four fermion interaction due torsion and general neutral current $J^\mu(p,p')$.}
%% \end{figure}
where $V_\mu(k) = \{A_\mu(k),Z^0_\mu(k)\}$ are the neutral gauge bosons to be considered and
\begin{align}
  J^\mu(p,p') &\equiv \bar{u}(p')\Bigg[\gamma^\mu\,F_V(k^2) +F_A(k^2)\gamma^\mu\gamma_5  \label{current} \\
    \nonumber
    &\quad + i\frac{\sigma^{\mu\nu}\,k_\nu}{2\,m_f}F_M(k^2) + F_D(k^2)\frac{1}{2m_f}\sigma^{\mu\nu}\gamma_5 k_\nu\Bigg]u(p)
\end{align}
is the more general neutral current constructed from Eq.~\eqref{lagvec} and \eqref{lagten}. $F_i(k)$, where $i=V,A,M,D$ denotes vector, axial, magnetic and dipole form factor respectively that plays an important role in precise measurments of radiative correction. In order to reproduce low energy regions with two possible neutral gauge boson exchange, the following condition for form factors must be satisfied. If one consider photon coupled to $J^\mu(p,p')$ in Eq.~\eqref{current}
\begin{align}
F_V^\gamma(0) &= Q_f,  \\
F_A^\gamma(0) &= 0, \\
F_M^\gamma(0) &= a_f^\gamma \equiv \frac{1}{2}\left(g_f-2\right),\\
F_D^\gamma(0) &= d_f^e\,\frac{2\,m_f}{e},
\end{align}
must be met, where $Q_f$, $a_f^\gamma$ and $d_f^e$ denotes unities of proton electric charge, the anomalous magnetic moment and electric dipole moment of the fermion $f$ respectively. Considering $Z^0$ coupled to $J^\mu(p,p')$ in Eq.~\eqref{current}
\begin{align}
F_V^{Z^0}(0) &= \frac{1}{2\,s_W\,c_W}\left(T_3^f - 2\,Q_f\,s_W^2\right), \\
F_A^{Z^0}(0) &=  \frac{1}{2\,s_W\,c_W}T_3^f, \\
F_M^{Z^0}(0) &= a_f^Z, \\
F_D^{Z^0}(0) &= d_f^w\,\frac{2\,m_f}{e}, 
\end{align}
must also satisfied, where $s_W(c_W) = \sin(\cos)\theta_W$ and $T_3^f$, $a_f^Z$ and $d_f^w$ denotes the third component of weak isospin, fermion weak magnetic moment and weak dipole moment of the fermion $f$ respectively. Comparing the general four fermionic interaction Lagrangians in Eq.~\eqref{lagvec} and \eqref{lagten} with Eq.~\eqref{4FI5D} coming from $D = 5$ torsionful manifold, one can identify
\begin{align}
\label{vectpar}
  V_V = 0 \espacio ; \espacio &A_V = 1 \espacio; \espacio \eta_V = +6 \\
\label{tenpar}
  V_T = 0 \espacio ; \espacio &A_T = 1 \espacio; \espacio \eta_T = +3
 \end{align}
The normalizations $g^2/4\pi = 1$ (if one consider different flavors on the loop i.e.: t-channel) and $g^2/2\pi = 1$ (if one consider the same flavors on the loop i.e.: s-channel) has been used. With the previous consideration, one could decompose the forms factors in their tree level value plus a contribution due radiative correction at one loop
\begin{align}
F_i^B(k^2) = F_i^{B\,\text{tree}} + \delta F_i^B(k^2),
\end{align}
where $i=V,A,M,D$ and $B=\gamma,Z^0$. Calculating this radiative corrections form factors $\delta F_i^B(k^2)$ is straighforward using the results obtained in Ref.~\cite{GonzalezGarcia:1998ay}. In the present work, $s$ and $t$-channels has been used with electrons in the final state and considering all possible particles running into the loop, giving for photon coupling
\begin{align}
 \delta F_V^\gamma(k^2) &= \frac{6}{\pi}\,\frac{k^2}{\Lambda^2}\,\ln\left(\frac{\Lambda^2}{\mu^2}\right), \\
 \delta F_A^\gamma(k^2) &= 0 , \\
 \delta F_M^\gamma(k^2) &= 0.220411\,\left(\frac{[\text{GeV}]}{\Lambda}\right)^2\,\ln\left(\frac{\Lambda^2}{\mu^2}\right), \\
 \delta F_D^\gamma(k^2) &= 0.
\end{align}
The same considerations has been used for $Z^0$ boson coupling with electrons in the final state. The following results has been obtained
\begin{align}
 \delta F_V^{Z^0}(k^2) &= -0.18084\,\frac{k^2}{\Lambda^2}\,\ln\left(\frac{\Lambda^2}{\mu^2}\right), \\
 \nonumber
 \delta F_A^{Z^0}(k^2) &= \left[9.56024\times10^{-2}\,\frac{k^2}{\Lambda^2} \right. \\
 & \left.+6.869\times10^4\,\left(\frac{[\text{GeV}]}{\Lambda}\right)^2\,\right]\,\ln\left(\frac{\Lambda^2}{\mu^2}\right), \\
 \delta F_M^{Z^0}(k^2) &= 7.9147\times10^{-2}\,\left(\frac{[\text{GeV}]}{\Lambda}\right)^2\,\ln\left(\frac{\Lambda^2}{\mu^2}\right), \\
 \delta F_D^{Z^0}(k^2) &= 0,
\end{align}
where $\mu$ denotes the scale involved in the process. If one consider all fermions in the loop, one could choose the mass of the heavier running fermion as the major mass scale involved in the process. In the following $\mu=m_t$ will be used.

An interesting thing happened by considering extra dimension scenario. If one focus only in $4$ dimensional scenario, the axial-tensor term in \eqref{4FI5D} disappear and the parameters on Eq.~\eqref{tenpar} vanishes, leading to no contribution to $\delta F_M^\gamma(k^2)$ and $\delta F_M^{Z^0}(k^2)$ in this theory. The appearance of magnetic forms factors in this theory is a purely effect of the fifth dimension. 


\section{\label{sec:constraints}Constraints from precision tests: $Z^0$ boson width decay}

In this section, our aim is to compare our theoretical results with experimental data. Using the obtained form factors, one could strongly constraint the scale for new physics coming from extra dimensions scenarios. This strength comes from comparing one-loop observables with precision tests of the SM. One of the most well known values, with excellent statistic and precision is the $Z^0$ width decay. Considering the most general current $J^\mu(p,p')$ in Eq.~\eqref{current}, the width decay of $Z^0$ into electrons can be decomposed
\begin{align}
\Gamma_{\text{teo}}\left(Z^0\rightarrow e^+\,e^-\right) &= \Gamma_{\text{SM}} + \delta\Gamma_{\text{4FI}}
\end{align}
where
\begin{align}
 \Gamma_{\text{SM}} &= \alpha M_Z\left[\frac{a_e^Z}{2s_Wc_W}(T_e^f - 2Q_f\,s_W^2)\right] 
\end{align}
is the the tree level contribution $Z^0$ boson width decay and
\begin{align}
\nonumber
  \delta\Gamma_{\text{4FI}} &= \frac{\alpha M_Z}{3\,s_Wc_W}\Bigg[(T_3^f - 2Q_f\,s_W^2+s_Wc_Wa_e^Z)\delta F_V^Z(M_Z^2)\\
  &\quad + T_3^f\delta F_A^Z(M_Z^2)\Bigg]
\end{align}
is the contribution due radiative correction at one loop in a effective theory coming from extra dimension scenario with four fermion interaction due torsionful manifold.

Considering electron and positron in the final state and evaluating $\alpha$, vector and axial form factors at $Z^0$ pole one obtain 
\begin{align}
\label{deltagammateo}
 \delta\Gamma_{\text{4FI}} &= -1.95982\times10^7\,[\text{MeV}]\left(\frac{[\text{GeV}]}{\Lambda}\right)^2\,\ln\left(\frac{\Lambda^2}{M_Z^2}\right).
\end{align}

The best updated values of $Z^0$ boson width decay into electron positron come from Ref.~\cite{Beringer:1900zz}
\begin{align}
\label{deltagammaexp}
 \Gamma_{\text{exp}} &= 83.984 \pm 0.086\;[\text{MeV}] \equiv \Gamma_{\text{SM}_\text{exp}} + \delta\Gamma_{\text{exp}}
\end{align}
and the effects of physics Beyond The Standard Model (BSM) has to be included (at least) in the experimental error $\delta\Gamma_{\text{exp}}$. Comparing Eq.~\eqref{deltagammateo} with Eq.~\eqref{deltagammaexp} one obtain a transcendental equation for the scale $\Lambda$ of physics BSM. Using the constraint $\delta\Gamma_{\text{exp}} \geq \delta\Gamma_{\text{4FI}}$ leads to
\begin{align}
\label{trascendental}
4.388\times10^{-9}\leq\left(\frac{[\text{GeV}]}{\Lambda}\right)^2\,\ln\left(\frac{\Lambda^2}{M_Z^2}\right)
\end{align}
that has solutions
\begin{eqnarray}
 \label{lowscalefromz}
 \Lambda_1^{Z} \leq&\; \pm0.09119\;&[\text{TeV}], \\
 \label{upscalefromz}
 \Lambda_2^{Z} \geq&\; \pm53.9364\;&[\text{TeV}].
\end{eqnarray}
Restricting only to positive energy scales, $\Lambda_1^Z$ is excluded by the experiments (see Ref.~\cite{Chatrchyan:2013muj}). Using this fact the scale for physics BSM coming from precision measuments of $Z^0$ boson width decay
\begin{align}
 \Lambda_Z &\geq 53.9364\;[\text{TeV}].
\end{align}
%% \begin{figure}[H]
%%  \includegraphics[scale=.34]{excl.eps}
%%  \caption{Exclusion limits for $\Lambda$ using the SM precision tests of $Z^0$ boson width decay. Blue curve line denotes $\delta\Gamma_{\text{4FI}}$ coming from torsionful extra dimension theory and the red straight one denotes $\delta\Gamma_{\text{exp}}$ obtained in SM precision tests. Shaded region denotes the error range of experimental data and is the allowed region for the solutions.}
%% \end{figure}

\begin{tikzpicture}
  \begin{axis}[
      title = Variation of $Z$ Width Decay,
      axis background/.style={
        shade,top color=gray!50,bottom color=white},
      xlabel = {Cut-off $\Lambda$},
      ylabel = {$\delta\Gamma_Z$},
      use units,
      x unit = {GeV}, y unit = {MeV},
      xmode = log,
    ]
    \addplot[ultra thick,
      red,
      samples = 301,
      domain=91.19:2e6,
    ] { 4.388e-9 };
    \addplot[ultra thick,
      blue,
      samples = 301,
      domain=91.19:2e6,
    ] { ln(x^2/(91.1876)^2)/x^2 };
    \coordinate (pt) at (axis cs: 2e5,1e-6);
  \end{axis}
  
  \node[pin=90:{
      \begin{tikzpicture}[baseline,trim axis left,trim axis right]
        \begin{axis}[
            axis background/.style={fill=white},
            footnotesize,
            grid = both,
            grid style = {black!20,dashed},
            xmode  = log,% ymode = log,
            xmin = 2e4, xmax = 2e6,
            ymin = 1e-11, ymax = 7e-9,
          ]
          
          \addplot[very thick,
            red,
            fill = red,
            fill opacity = .2,
            domain=20000:2100000,
            samples=301,
          ] { 4.388e-9 } |- (axis cs:20000,-1e-9);
          \addplot[very thick,
            blue,
            domain=20000:2100000,
            samples=301,
          ] { ln(x^2/(91.1876)^2)/x^2 };
        \end{axis}
      \end{tikzpicture}%
  }] at (pt) {};
\end{tikzpicture}

This limits are consistents with Ref.~\cite{Chang:2000yw} but are more stronger than $\Lambda \geq 28\,[\text{TeV}]$ found in the context of Universal Extra Dimensions (UED) studied in the referred work.

Now comparing with the effective coupling coming from large extra dimension scenario, one find 
\begin{align}
 \frac{\kappa_{\text{eff}}^2}{32} \longleftrightarrow \frac{1}{\Lambda^2}
\end{align}
this comparison leads to the constraint
\begin{align}
\label{parconst}
\left(\frac{M_*^3}{k}\right)^{\frac{1}{2}}\geq \left\{ \begin{matrix}
                                                        5.0801\times10^{-7}\;[\text{TeV}]& \text{if} & c_i\simeq0 \\
                                                        7.5874\times10^{-2}\; [\text{TeV}]& \text{if} & c_i\simeq1/2 \\
                                                        1.9462\;[\text{TeV}]& \text{if} & c_i\simeq1 \\
                                                       \end{matrix} \right.
\end{align}


% \subsection{Electron $g-2$}
% 
% Electron $g-2$ is also a well-known observable with high precision measurement and is related with magnetic form factor via
% \begin{align}
% \label{gm2}
% \delta F_M^\gamma(k^2=0) &= a_e^\gamma \equiv \frac{1}{2}\left(g_e-2\right)
% \end{align}
% where $g_e$ is the anomalous magnetic moment of the electron. Considering SM and torsion contributions, Eq.~\eqref{gm2} can be decomposed
% \begin{align}
% a_e^\gamma &= a_{\text{SM}}^\gamma + \delta a_{e\,\text{4FI}}^\gamma
% \end{align}
% where $\delta F_M^\gamma(k^2=0) = \delta a_{e\,\text{4FI}}^\gamma$ denotes the contribution due radiative correction involving theory with four fermion interaction due torsion. 
% 
% Considering electrons in the t-channel of Eq.~\eqref{feyndiagram} with all possible fermions running into the loop, the contribution due torsion to $a_e^\gamma$ gives
% \begin{align}
% \label{aeteo}
%  \delta a_e^\gamma &= 0.22041\,\left(\frac{\text{GeV}}{\Lambda}\right)^2\,\ln\left(\frac{\Lambda^2}{\mu^2}\right).
% \end{align}
% 
% The best updated data of electron $g-2$ are obtained in Ref.~\cite{Hanneke:2008tm} and also in Ref.~\cite{Beringer:1900zz}
% \begin{align}
% \label{aeexp}
%  a_e^\gamma &= (1159.65218076\pm0.00000028)\times10^{-6} 
% \end{align}
% where one can decompose $a_e^\gamma = a_{e\,\text{SM}_\text{exp}}^\gamma + \delta a_{e\,\text{exp}}^\gamma$. Similar to $Z^0$ boson width decay, physics BSM has to be included (at least) in the experimental error $\delta a_{e\,\text{exp}}^\gamma$. Comparing Eq.~\eqref{aeteo} with Eq.~\eqref{aeexp} one obtain a transcendental equation for the scale of physics BSM. Using the constraint $\delta a_{e\,\text{exp}}^\gamma\geq\delta a_{e\,\text{4FI}}^\gamma$ leads to
% \begin{align}
% 1.22498\times10^{-12} \leq \left(\frac{[\text{GeV}]}{\Lambda}\right)^2\,\ln\left(\frac{\Lambda^2}{\mu^2}\right) 
% \end{align}
% that has solutions
% \begin{eqnarray}
%  \label{lowscalefromgm2}
%  \Lambda_1^{e} \leq&\; \pm0.17307\;&[\text{TeV}], \\
%  \label{upscalefromgm2}
%  \Lambda_2^{e} \geq&\; \pm4.053\times10^3\;&[\text{TeV}].
% \end{eqnarray}
% 
% Restricting only to positive scales of energy and excluding $\Lambda\leq0.17307\;[\text{TeV}]$ from Ref.~\cite{Chatrchyan:2013muj} one find a scale for physics BSM coming from electron $g-2$ experiment 
% \begin{align}
%  \Lambda_e &\geq 4.053\times10^3\;[\text{TeV}].
% \end{align}
% 
% Performing the same analysis on $Z^0$ boson width decay but in this case using electron $g-2$ precision measurements, one gets
% \begin{align}
% \left(\frac{M_*^3}{k}\right)^{\frac{1}{2}}\geq \left\{ \begin{matrix}
%                                                         2.1595\times10^{-4}\;[\text{TeV}]& \text{if} & c_i\simeq0 \\
%                                                         0.3225\times10^{2}\; [\text{TeV}]& \text{if} & c_i\simeq1/2 \\
%                                                         0.8273\times10^{3}\;[\text{TeV}]& \text{if} & c_i\simeq1 \\
%                                                        \end{matrix} \right.
% \end{align}

\section{\label{sec:conclusions}Conclusions}

In the present work a more general formulation of Gravity has been consider: the Cartan-Einstein Formalism. This framework gives rise to a contact four fermion interaction from the equations of motion, that is highly suppressed by the inverse of the squared Planck mass in $4$-dimensions. 

In order to deal with this, an scenario without this hierarchy between gravitational and SM interactions has been used: RS model \cite{Randall:1999ee}. This model suggests the existence of one large extra dimension and the hierarchy between the fundamental ($5$-dimensional) and the effective ($4$-dimensional) Planck mass is an exponential function. However, in order to deal with $4$-dimensional observables, this large extra dimension is compactified on an orbifold $S^1/\mathbb{Z}_2$ of radius $R$. This compactification leads to an effective theory in $4$-dimensions with a four fermion interaction due torsion. 

Considering this four fermion interaction in RS scenario, it can be decomposed in two terms: an axial-vector and axial-tensor interaction. By geometrical reasons of the present model this last term vanishes and leads only to the axial-vector one. One of the phenomenological implication of this absence is that the torsion contributions to observables like leptonic anomalous magnetic moment are forbidden. 

Although this axial-tensor interaction does not appears in the effective theory one can explore axial-vector mediated observables as $Z^0$ boson width decay at one-loop level. In Sec.~\ref{sec:oneloop} form factors are obtained from neutral bosons exchange processes. Using these form factors and the calculation of $Z^0$ boson width decay, the scale for new physics coming from torsionful extra dimension scenario has been achieved. The use of SM presicion tests data leads to a scale for new physics $ \Lambda \geq 53.9364\;[\text{TeV}]$ which constraints strongly the parameters of this theory, for different fermion localization values, as Eq.~\eqref{parconst} shown.

\section*{Acknowledgement}

C.C. would like to thank A. Toloza, A. Cisterna for fruitful discussions and kindness. Also thanks to C. Ayala for his help with the figures through this article. This work was partially supported by Conicyt (Chile) under Grant No. 21130179.

\appendix

\section{\label{sec:cliff}Clifford Algebras}

\subsection{Even-dimensional Clifford algebra}
The  Clifford algebra on a general even $D$-dimensional, with $D=2m$ is  expanded by
\begin{equation}
  \Gamma^A = \left\{  \1,\gamma^{\hat{a}_1},\gamma^{\hat{a}_1\hat{a}_2},\ldots,\gamma^{\hat{a}_1\ldots \hat{a}_D} \right\}
\end{equation}
where  $\gamma^{\hat{a}_1\ldots \hat{a}_r} = \gamma^{[\hat{a}_1}\ldots\gamma^{\hat{a}_r]}$ has been defined. %% Now, one can define the highest rank Clifford algebra element, 
On  even dimensions, a chirality element can be defined by
\begin{equation}
  \label{gs}
  \gs = \frac{1}{D!}\imath^{-(m+1)}\epsilon_{\hat{a}_1\ldots \hat{a}_D}\gamma^{\hat{a}_1\ldots \hat{a}_D}
\end{equation}
where the convention $\epsilon_{012\ldots(D-1)} = 1$ has been used. The matrix representation of $\gs$ is  known in physics literature as $\gamma^{D+1}$; for example, in $D=4$ is the familiar $\gamma^5$ matrix. Additionally, the chiral element satisfies $\Tr{\left[\gs\right]}=0$ and $\left(\gs\right)^2 = +\1$. 

From  the definition  of $\gamma^{\hat{a}_1\ldots\hat{a}_r}$ and $\gs$ it  can be shown that
\begin{equation}
  \gamma^{\hat{a}_1\ldots \hat{a}_r} = \frac{(-\imath)^{m+1}}{(D-r)!}\epsilon^{\hat{a}_r \hat{a}_{r-1}\ldots \hat{a}_1 \hat{b}_1\ldots \hat{b}_{D-r}}\gamma_{\hat{b}_1\ldots \hat{b}_D}\gs .
\end{equation}

Then, %%For example, let's consider $D=2m$ dimensional space, and 
for $r=3$ one obtains
\begin{align}
  \gamma^{\hat{a}\hat{b}\hat{c}} &= \frac{(-\imath)^{m+1}}{(D-3)!}\epsilon^{\hat{a}\hat{b}\hat{c}\,\hat{a}_1\ldots \hat{a}_{D-3}}\gamma_{\hat{a}_1\ldots \hat{a}_{D-3}}\gs, \\
  \gamma_{\hat{a}\hat{b}\hat{c}} &= \frac{(-\imath)^{m+1}}{(D-3)!}\epsilon_{\hat{a}\hat{b}\hat{c}\,\hat{a}_1\ldots \hat{a}_{D-3}}\gamma^{\hat{a}_1\ldots \hat{a}_{D-3}}\gs. 
\end{align}
Contracting these antisymmetric tensors, one gets
\begin{align}
  \nonumber
  (\gamma^{\hat{a}\hat{b}\hat{c}})(\gamma_{\hat{a}\hat{b}\hat{c}}) &= \frac{1}{(D-3)!^2}\epsilon^{\hat{a}\hat{b}\hat{c}\hat{a}_1\ldots \hat{a}_{D-3}}\epsilon_{\hat{a}\hat{b}\hat{c}\hat{b}_1\ldots \hat{b}_{D-3}}\notag\\
  &\qquad \gamma_{\hat{a}_1\ldots \hat{a}_{D-3}}\gs\gamma^{\hat{b}_1\ldots \hat{b}_{D-3}}\gs
\end{align}
and using  the identity 
\begin{align}
  \epsilon^{\hat{a}\hat{b}\hat{c}\,\hat{a}_1\ldots \hat{a}_{D-3}}\epsilon_{\hat{a}\hat{b}\hat{c}\,\hat{b}_1\ldots \hat{b}_{D-3}} = -3!\delta^{\hat{a}_1}_{[\hat{b}_1}\ldots \delta^{\hat{a}_{D-3}}_{\hat{b}_{D-3}]},
\end{align}
%% Contracting this last generalized antisymmetric Kronecker delta, and reordering antisymmetric gamma product, one obtain an extra $(D-3)!$, that canceled one of this factor in the denominator. With this, o
One finally gets a general relation for even dimension
\begin{equation}
  (\gamma^{\hat{a}\hat{b}\hat{c}})(\gamma_{\hat{a}\hat{b}\hat{c}}) = (-1)^m \frac{3!}{(D-3)!}(\gamma_{\hat{b}_1\ldots \hat{b}_{D-3}}\gs)(\gamma^{\hat{b}_1\ldots \hat{b}_{D-3}}\gs).
\end{equation}

%% In the particular  case of $D=4$ ($m=2$)
Particularly, in four dimensions one can write
\begin{equation}
  (\gamma^{\hat{a}\hat{b}\hat{c}})(\gamma_{\hat{a}\hat{b}\hat{c}}) =  6(\gamma_{\hat{d}}\gs)(\gamma^{\hat{d}}\gs)
\end{equation}
%% where $a$ index without hat, denotes the $D=2m$ tangent space coordinates and the definition for $\gs$ is in Eq.~\eqref{gs}

\subsection{Odd-dimensional Clifford algebra}
The main idea is construct the $D=2m+1$ dimensional Clifford algebra, using the $D=2m$ dimensional one, and the highest rank Clifford algebra element $\gs$. Expanding the basis for $D=2m+1$,
\begin{equation}
  \ghu{a} = \left(\gamma^a,\gs\right)
\end{equation}
where the notation without hat for previous even dimensions has been used. In this case, one found
\begin{equation}
  \gamma^{\hat{a}_1\ldots \hat{a}_r} = \frac{i^{m+1}}{(D-r)!}\epsilon^{\hat{a}_1\ldots\hat{a}_r\hat{a}_{r+1}\ldots\hat{a}_D}\gamma_{\hat{a}_D\ldots\hat{a}_{r+1}}.
\end{equation}
Let consider the particular case with $r=3$, with this
\begin{align}
  \ghhhu{a}{b}{c} &= \frac{i^{m+1}}{(D-3)!}\epsilon^{\hat{a}\hat{b}\hat{c}\hat{c}_4\ldots\hat{c}_D}\gamma_{\hat{c}_D\ldots\hat{c}_4} \\
  \ghhhd{a}{b}{c} &= \frac{i^{m+1}}{(D-3)!}\epsilon_{\hat{a}\hat{b}\hat{c}\hat{d}_4\ldots\hat{d}_D}\gamma^{\hat{d}_D\ldots\hat{d}_4} \\
\end{align}
with this, and contracting both antisymmetric product of gamma matrices, one found
\begin{align}
  \ghhhu{a}{b}{c}\ghhhd{a}{b}{c} &= \frac{i^{(2m+2)}}{\left((D-3)!\right)^2}\epsilon^{\hat{a}\hat{b}\hat{c}\hat{c}_4\ldots\hat{c}_D}\epsilon_{\hat{a}\hat{b}\hat{c}\hat{d}_4\ldots\hat{d}_D}\gamma_{\hat{c}_D\ldots\hat{c}_4}\gamma^{\hat{d}_D\ldots\hat{d}_4}
\end{align}
using the identity 
\begin{align}
  \epsilon^{\hat{a}\hat{b}\hat{c}\hat{c}_4\ldots\hat{c}_D}\epsilon_{\hat{a}\hat{b}\hat{c}\hat{d}_4\ldots\hat{d}_D} = -3!\delta^{\hat{c}_4}_{[\hat{d}_4}\ldots\delta^{\hat{c}_D}_{\hat{d}_D]}
\end{align}
and contracting, and reordering the antisymmetric product of gamma matrices, one gets (like in even case), a factor of $(D-3)!$ that canceled one factor on the denominator, with this, one gets the general case for odd dimension
\begin{equation}
  \ghhhu{a}{b}{c}\ghhhd{a}{b}{c} = \frac{i^{(2m+2)}}{(D-3)!}(-3!)\gamma_{\hat{c}_4\ldots\hat{c}_D}. \gamma^{\hat{c}_4\ldots\hat{c}_D}
\end{equation}

For example,  taking $D=5$
\begin{align}
  \ghhhu{a}{b}{c}\ghhhd{a}{b}{c} &= +\frac{3!}{2!}\gamma_{\hat{c}\hat{d}}\gamma^{\hat{c}\hat{d}} \\
  &= 3\left(\gamma_{ab}\gamma^{ab} + \gamma_{a*}\gamma^{a*} + \gamma_{*a}\gamma^{*a}\right)\\
  &= 3\left(\gamma_{ab}\gamma^{ab} + 2\gamma_{a*}\gamma^{a*}\right)
\end{align}
one can multiply by one $\left(\gs\right)^2 = \1$, the tensorial part and reordering one gets
\begin{equation}
  \ghhhu{a}{b}{c}\ghhhd{a}{b}{c} = 6\gamma_a\gs\gamma^a\gs + 3\gamma_{ab}\gs\gamma^{ab}\gs
\end{equation}



\bibliographystyle{unsrt}
\bibliography{bibliography}

\end{document}
