\section{\label{sec:oneloop}One loop calculations and form factors}


In this section, the effects of curvature in the effective theory in $4$-dimensions will be ignored, by the fact that the Universe is essentially flat as Ref.~\cite{Larson:2010gs} indicates. This assumption is also based in our comparision of torsion effects with particle accelerators data (the predominant forces in these experiments becames from the SM interactions, and the curvature effects are negligible). Obviously, this assumption is not valid anymore where curvature effects can not be droped, i.e.: near to a black hole or neutron star. This consideration has been used before in Ref.~\cite{Carroll:1994dq,Belyaev:1998ax,Kostelecky:2007kx}. In CEF the metric and connection are independent, this condition allows to curvature and torsion be independent too (this can be understood from Eq.~\eqref{cartantorsion} and Eq.~\eqref{cartancurvature}). There exist manifolds with torsion and no curvature, where teleparallel gravity relies (for further reading, see Ref.~\cite{Arcos:2005ec}). This special kind of manifolds are called Weitzenb\"ock manifolds.

Our interest by now, is try to extract information of the contribution of torsion to one loop form factors. Now, considering that $SU(2)_L\otimes U(1)_Y$ gauge sector of the SM is torsion free, the only effect of torsion is through the four fermion contact term in Eq.~\eqref{4FI5D}. Using this kind of interaction, our aim is to do one loop calculation in this theory. 

The process to be calculated in this section can be extracted from the general four femionic interaction Lagrangian used in Ref.~\cite{GonzalezGarcia:1998ay} plus one neutral gauge boson exchange. The relevant Lagrangian used by Gonzalez-Garc\'ia, Gusso and Novaes in the previous reference is
 \begin{align}
  \nonumber
   \Lag_{\text{V}} &= \eta_V\,\frac{g^2}{\Lambda^2}\left[\psi_r\gamma_\mu\left(V_V - A_V\gamma^*\right)\psi_r\right] \\ 
  \label{lagvec}
  &\espacio\espacio\espacio\times \left[\psi_s\gamma^\mu\left(V_V - A_V\gamma^*\right)\psi_s\right],
 \end{align}
where $r$ and $s$ denotes flavor indices as in the previous sections. The most general one loop calculation in this theory, can be build with the previous four fermion interaction Lagrangian and gauge boson coupled to fermions
\begin{align}
\label{feyndiagram}
  \begin{tikzpicture}[thick,baseline=(current  bounding  box.center)]
    \coordinate (V) at (0,0);
    \node[circle,draw=black,shade,minimum size=.6cm]  at (V)  {};
    \draw[boson] (-2,0) node[anchor=south] {$V_\mu(k)$} -- (180:3mm);
    \draw[directed] (1,-1) node[anchor=west] {$f(p)$}  -- (-45:3mm);
    \draw[directed] (45:3mm) -- (1,1) node[anchor=west] {$f(p')$};
  \end{tikzpicture}
  =\imath e \, V_\mu(k) J^\mu(p,p')
\end{align}
%% \begin{figure}[H]
%% \scalebox{1.5}[1.5]{\begin{tikzpicture}
%%  \draw[boson,thick] (-5,0) -- (-3.3,0);
%%  \node[above,scale=.7] at (-4,.2) {$Z^0_\mu(k),A_\mu(k)$};
%%  \draw[thick,directed] (-2,-1) -- (-3,0);
%%  \node[right,scale=.7] at (-2.3,-.5) {$f(p)$};
%%  \draw[thick,directed] (-3,0) -- (-2,1);
%%  \node[right,scale=.7] at (-2.3,.5) {$f(p')$};
%%  \filldraw[shade] (-3,0) circle (.3);
%%  \node[scale=.6] at (-2,0) {$=$};
%%  \node[scale=.6] at (-.8,0) {$(i\,e)V_\mu(k) J^\mu(p,p')$};
%% \end{tikzpicture}}
%% \caption{Feynman diagram for one loop process including gauge coupling, four fermion interaction due torsion and general neutral current $J^\mu(p,p')$.}
%% \end{figure}
where $V_\mu(k) = \{A_\mu(k),Z^0_\mu(k)\}$ are the neutral gauge bosons to be considered and
\begin{widetext}
\begin{align}
 \label{current}
  J^\mu(p,p') &\equiv \bar{u}(p')\Bigg[\gamma^\mu\,F_V(k^2) +F_A(k^2)\gamma^\mu\gamma^* + i\frac{\sigma^{\mu\nu}\,k_\nu}{2\,m_f}F_M(k^2) + F_D(k^2)\frac{1}{2m_f}\sigma^{\mu\nu}\gamma^* k_\nu\Bigg]u(p)
\end{align}
\end{widetext}
is the more general neutral current constructed from Eq.~\eqref{lagvec}. $F_i(k)$, where $i=V,A,M,D$, denotes vector, axial, magnetic and dipole form factor respectively that plays an important role in precise measurments of radiative correction. In order to reproduce low energy regions with two possible neutral gauge boson exchange, the following condition for form factors must be satisfied. If one consider photon coupled to $J^\mu(p,p')$ in Eq.~\eqref{current}
\begin{align}
F_V^\gamma(0) &= Q_f,  \\
F_A^\gamma(0) &= 0, \\
F_M^\gamma(0) &= a_f^\gamma \equiv \frac{1}{2}\left(g_f-2\right),\\
F_D^\gamma(0) &= d_f^e\,\frac{2\,m_f}{e},
\end{align}
must be met, where $Q_f$, $a_f^\gamma$ and $d_f^e$ denotes unities of proton electric charge, the anomalous magnetic moment and electric dipole moment of the fermion $f$ respectively. Considering $Z^0$ coupled to $J^\mu(p,p')$ in Eq.~\eqref{current}
\begin{align}
F_V^Z(0) &= \frac{1}{2\,s_W\,c_W}\left(T_3^f - 2\,Q_f\,s_W^2\right), \\
F_A^Z(0) &=  \frac{1}{2\,s_W\,c_W}T_3^f, \\
F_M^Z(0) &= a_f^Z, \\
F_D^Z(0) &= d_f^w\,\frac{2\,m_f}{e}, 
\end{align}
must also satisfied, where $s_W(c_W) = \sin(\cos)\theta_W$ and $T_3^f$, $a_f^Z$ and $d_f^w$ denotes the third component of weak isospin, fermion weak magnetic moment and weak dipole moment of the fermion $f$ respectively. Comparing the general four fermionic interaction Lagrangian in Eq.~\eqref{lagvec} with Eq.~\eqref{4FI5D} coming from $D = 5$ torsionful manifold, one can identify
\begin{align}
\label{vectpar}
  V_V = 0 \espacio ; \espacio &A_V = 1 \espacio; \espacio \eta_V = +6 
 \end{align}
The normalizations $g^2/4\pi = 1$ (if one consider different flavors on the loop i.e.: t-channel) and $g^2/2\pi = 1$ (if one consider the same flavors on the loop i.e.: s-channel) has been used. With the previous consideration, one could decompose the forms factors in their tree level value plus a contribution due radiative correction at one loop
\begin{align}
F_i^B(k^2) = F_i^{B\,\text{tree}} + \delta F_i^B(k^2),
\end{align}
where $i=V,A,M,D$ and $B=\gamma,Z^0$. Calculating this radiative corrections form factors $\delta F_i^B(k^2)$ is straighforward using the results obtained in Ref.~\cite{GonzalezGarcia:1998ay}. In the present work, $s$ and $t$-channels has been used with electrons in the final state and considering all possible particles running into the loop, giving for photon coupling
\begin{align}
 \delta F_V^\gamma(k^2) &= \frac{6}{\pi}\,\frac{k^2}{\Lambda^2}\,\ln\left(\frac{\Lambda^2}{\mu^2}\right), \\
 \delta F_A^\gamma(k^2) &= 0 , \\
 \delta F_M^\gamma(k^2) &= 0 , \\
 \delta F_D^\gamma(k^2) &= 0.
\end{align}
The same considerations has been used for $Z^0$ boson coupling with electrons in the final state. The following results has been obtained
\begin{align}
 \delta F_V^Z(k^2) &= -0.18084\,\frac{k^2}{\Lambda^2}\,\ln\left(\frac{\Lambda^2}{\mu^2}\right), \\
 \nonumber
 \delta F_A^Z(k^2) &= \left[9.56024\times10^{-2}\,\frac{k^2}{\Lambda^2} \right. \\
 & \left.+6.869\times10^4\,\left(\frac{[\text{GeV}]}{\Lambda}\right)^2\,\right]\,\ln\left(\frac{\Lambda^2}{\mu^2}\right), \\
 \delta F_M^Z(k^2) &= 0 , \\
 \delta F_D^Z(k^2) &= 0 ,
\end{align}
where $\mu$ denotes the scale involved in the process (i.e.: for $Z^0$ width decay $\mu=m_Z$ will be used).

The absence of axial-tensor current interaction in Eq.~\eqref{4FI5D} leads to zero contribution to magnetic form factor $\delta F_M^B$. This fact constraints no contribution to fermion anomalous magnetic moment at one-loop in a effective theory coming from $5$-dimensional RS set up.

