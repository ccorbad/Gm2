\section{\label{sec:conclusions}Conclusions}

In the present work a more general formulation of Gravity has been consider: the Cartan-Einstein Formalism. This framework gives rise to a contact four fermion interaction from the equations of motion, that is highly suppressed by the inverse of the squared Planck mass in $4$-dimensions. 

In order to deal with this, an scenario without this hierarchy between gravitational and SM interactions has been used: RS model \cite{Randall:1999ee}. This model suggests the existence of one large extra dimension. However, in order to deal with $4$-dimensional observables, this large extra dimension is compactified on an orbifold $S^1/\mathbb{Z}_2$ of radius $R$. This compactification leads to an effective theory in $4$-dimensions with a four fermion interaction due torsion. The relation between the fundamental ($M_*$) and the effective ($M_{\text{Pl}}$) Planck mass, appears through the dimensional reduction as an exponential function.  

Considering this four fermion interaction in RS scenario, it can be decomposed in two terms: an axial-vector and axial-tensor interaction. By geometrical reasons of the present model this last term vanishes and leads only to the axial-vector one. One of the phenomenological implication of this absence is that, in RS model with one large extra dimension, torsion contributions to observables like leptonic anomalous magnetic moment are forbidden. 

Although this axial-tensor interaction does not appears in the effective theory one can explore axial-vector mediated observables as $Z^0$ boson width decay at one-loop level. In Sec.~\ref{sec:oneloop} form factors are obtained from neutral bosons exchange processes. Using these form factors and the calculation of $Z^0$ boson width decay, the scale for new physics coming from torsionful extra dimension scenario has been achieved. The use of SM presicion tests data leads to a scale for new physics $ \Lambda \geq 53.9364\;[\text{TeV}]$ which constraints strongly the parameters of this theory, for different fermion localization values, as Eq.~\eqref{parconst} shown.
