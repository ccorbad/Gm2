\documentclass[twocolumn,showpacs,showkeys,prd,superscriptaddress]{revtex4-1}

\usepackage{float}
\usepackage{subfigure}
\usepackage{siunitx}
\sisetup{
  separate-uncertainty = true
}
\usepackage{ragged2e}

%%%%%%%%% CORRECTIONS %%%%%%%%%
\usepackage{ulem}
\newcommand\pro[1]{{\color{blue}#1}}
\newcommand\out[1]{{\color{red}\sout{#1}}}
%%%%%%%%%%%%%%%%%%%%%%%%%%%%%%%

%---------Packages-------------
\usepackage{amsmath,amssymb,amsfonts,dsfont,mathrsfs,amsthm}
\usepackage{graphicx}
\usepackage{bbold}
\usepackage{slashed}
\usepackage{centernot}
\usepackage{hyperref}
\usepackage{lmodern}
\usepackage{xcolor}
\usepackage{comment}
\usepackage{epstopdf}
\hypersetup{linktocpage,colorlinks=true,urlcolor=blue,linkcolor=blue,citecolor=red}
\usepackage{feynmf}
\usepackage{array}

%---------Theorems------------
\newtheorem{Def}{Definition}
\newtheorem{Thm}{Theorem}
\newtheorem{Lem}{Lemma}
\newtheorem{Pos}{Postulate}
\newtheorem{Exa}{Example}
\newtheorem{Cor}{Corrolary}
\newtheorem{Pro}{Proposition}

%-------New Commands--------
\newcommand{\qd}{\textquestiondown}
\newcommand{\titulo}[1]{\Huge\textbf{#1}}
\newcommand{\lagrange}[1]{\frac{\partial\Lag}{\partial #1} - \frac{d}{dt}\frac{\partial\Lag}{\partial\dot{#1}} = 0}
\newcommand{\parcial}[2]{\frac{\partial #1}{\partial #2}}
\newcommand{\parciald}[2]{\frac{\delta #1}{\delta #2}}
\newcommand{\lagranged}[1]{\frac{\delta\Lag}{\delta #1} - \partial_\mu\frac{\delta\Lag}{\delta\left(\partial_\mu #1\right)} = 0}
\newcommand{\espacio}{\,\,\,\,\,}
\newcommand{\A}{\mathcal{A}} 
\newcommand{\abs}[1]{\left|#1\right|}
\newcommand{\C}{\mathbb{C}}
\newcommand{\bboxed}[1]{{\color{red}{\boxed{\boxed{\textcolor{black}{#1}}}}}}
\newcommand{\D}{\mathscr{D}}
\newcommand{\J}{\mathscr{J}}
\newcommand{\Lag}{\mathscr{L}}
\newcommand{\Lap}{\nabla^2}
\newcommand{\ket}[1]{\left.\left|#1\right.\right>}
\newcommand{\bra}[1]{\left.\left<#1\right.\right|}
\newcommand{\bkt}[3]{\left<#1\left|#2\right|#3\right>}
\newcommand{\bk}[2]{\left<#1\left|#2\right.\right>}
\newcommand{\comm}[2]{\left[#1,#2\right]}
\newcommand{\anticomm}[2]{\left\{#1,#2\right\}}
\newcommand{\vev}[1]{\ensuremath{\left<#1\right>}}
\newcommand{\uf}[2]{\ensuremath{u\(\vec{#1},#2\)}}
\newcommand{\ufb}[2]{\ensuremath{\bar{u}\(\vec{#1},#2\)}}
\newcommand{\vf}[2]{\ensuremath{v\(\vec{#1},#2\)}}
\newcommand{\vfb}[2]{\ensuremath{\bar{v}\(\vec{#1},#2\)}}
\newcommand{\ann}[3]{\ensuremath{#1\(\vec{#2},#3\)}}
\newcommand{\cre}[3]{\ensuremath{#1^\dag\(\vec{#2},#3\)}}
\newcommand{\vif}[1]{{\bf{e}}^{{#1}}}
\newcommand{\vifh}{\hat{{\bf{e}}}}
\newcommand{\vifhn}[2]{\hat{{\bf{e}}}^{\hat{#1}_{#2}}}
\newcommand{\etah}[2]{{\eta}_{\hat{#1}\hat{#2}}}
\newcommand{\etahn}[4]{{\eta}_{\hat{#1}_{#2}\hat{#3}_{#4}}}
\newcommand{\etahnu}[4]{{\eta}^{\hat{#1}_{#2}\hat{#3}_{#4}}}
\newcommand{\dduhn}[4]{\delta_{\hat{#1}_{#2}}^{\hat{#3}_{#4}}}
\newcommand{\vih}{\hat{e}}
\newcommand{\wfh}{\hat{{\boldsymbol{\omega}}}}
\newcommand{\wfhn}[4]{\hat{{\boldsymbol{\omega}}}^{\hat{#1}_{#2}\,\hat{#3}_{#4}}}
\newcommand{\wfudhn}[4]{\hat{{\boldsymbol{\omega}}}^{\hat{#1}_{#2}}{}_{\hat{#3}_{#4}}}
\newcommand{\wfhfree}{\hat{\mathring{{\boldsymbol{\omega}}}}}
\newcommand{\wfhnfree}[4]{\hat{\mathring{{\boldsymbol{\omega}}}}^{\hat{#1}_{#2}\hat{#3}_{#4}}}
\newcommand{\wfhnudfree}[4]{\hat{\mathring{{\boldsymbol{\omega}}}}^{\hat{#1}_{#2}}{}_{\hat{#3}_{#4}}}
\newcommand{\Rf}[2]{{\boldsymbol{\mathcal{R}}}^{#1 #2}}
\newcommand{\Rfhn}[4]{\hat{{\boldsymbol{\mathcal{R}}}}^{\hat{#1}_{#2}\hat{#3}_{#4}}}
\newcommand{\Rfhnfree}[4]{\hat{\mathring{{\boldsymbol{\mathcal{R}}}}}^{\hat{#1}_{#2}\hat{#3}_{#4}}}
\newcommand{\Tf}[1]{{\boldsymbol{\mathcal{T}}}^{#1}}
\newcommand{\Th}{\hat{\mathcal{T}}}
\newcommand{\Tfhn}[2]{\hat{{\boldsymbol{\mathcal{T}}}}^{\hat{#1}_{#2}}}
\newcommand{\hodge}{\star}
\newcommand{\K}{\mathcal{K}}
\newcommand{\Ma}{\mathcal{M}}
\newcommand{\R}{\mathcal{R}}
\newcommand{\Rh}{\hat{{\mathcal{R}}}}
\newcommand{\etahd}[2]{\hat{\eta}_{\hat{#1}\hat{#2}}}
\newcommand{\etahu}[2]{\hat{\eta}^{\hat{#1}\hat{#2}}}
\newcommand{\Rfh}{\hat{{\boldsymbol{\mathcal{R}}}}}
\newcommand{\Kf}{{\boldsymbol{\mathcal{K}}}}
\newcommand{\Kfhn}[4]{\hat{{\boldsymbol{\mathcal{K}}}}^{\hat{#1}_{#2}\hat{#3}_{#4}}}
\newcommand{\Kfhnud}[4]{\hat{{\boldsymbol{\mathcal{K}}}}^{\hat{#1}_{#2}}{}_{\hat{#3}_{#4}}}
\newcommand{\Kfuu}[2]{{\boldsymbol{\mathcal{K}}}^{#1 #2}}
\newcommand{\Kfdd}[2]{{\boldsymbol{\mathcal{K}}}_{#1 #2}}
\newcommand{\Kfud}[2]{{\boldsymbol{\mathcal{K}}}^{#1}{}_{#2}}
\newcommand{\wf}{{\boldsymbol{\omega}}}
\newcommand{\wfb}{\bar{{\boldsymbol{\omega}}}}
\newcommand{\Df}{{\boldsymbol{D}}}
\newcommand{\df}{{\boldsymbol{d}}}
\newcommand{\gf}{{\boldsymbol{\gamma}}}
\newcommand{\Af}{{\boldsymbol{A}}}
\newcommand{\T}{\mathcal{T}}
\newcommand{\free}[1]{\mathring{#1}}
\newcommand{\ghu}[1]{\gamma^{\hat{#1}}}
\newcommand{\ghhhu}[3]{\gamma^{\hat{#1}\hat{#2}\hat{#3}}}
\newcommand{\ghhhd}[3]{\gamma_{\hat{#1}\hat{#2}\hat{#3}}}
\newcommand{\1}{\mathbb{1}}
\newcommand{\gs}{\gamma^{*}}
\newcommand{\form}[1]{{\boldsymbol{#1}}}

%--------------Operators------------
\newcommand{\diag}{\operatorname{diag}}
\newcommand{\tr}{\operatorname{tr}}
\newcommand{\Tr}{\operatorname{Tr}}
\newcommand{\Ker}{\operatorname{Ker}}
\renewcommand{\Im}{\operatorname{Im}}
\newcommand{\sgn}{\operatorname{sgn}}
\newcommand{\Ln}{\operatorname{Ln}}
\newcommand{\Ei}{\operatorname{Ei}}
\newcommand{\csch}{\operatorname{csch}}
\newcommand{\arcsinh}{\operatorname{arcsinh}}

\newcommand\dv[1][]{\ensuremath{\mathrm{d}V_{\! #1}}}

\usepackage{tikz}
\usetikzlibrary{arrows,shapes,positioning}
\usetikzlibrary{decorations.markings}
\usetikzlibrary{decorations.pathreplacing}
\usetikzlibrary{decorations.pathmorphing}
\tikzstyle directed=[postaction={decorate,decoration={markings,mark=at position .5 with {\arrow{stealth}}}}]
\tikzset{%
  cross/.style={path picture={ 
      \draw[black]
      (path picture bounding box.south east) -- (path picture bounding box.north west) 
      (path picture bounding box.south west) -- (path picture bounding box.north east);
}}}
\tikzset{boson/.style={decorate,decoration={snake}}}

\usepgfplotslibrary{fillbetween}


\begin{document}
%\title{Torsional Contribution to One-Loop Observables in an Effective Theory Coming From Extra Dimensions}
\title{Torsion in Extra Dimensions and One-Loop Observables}

\author{Oscar \surname{Castillo-Felisola}}
\email{o.castillo.felisola@gmail.com} 
\affiliation{Departamento de F\'\i sica, Universidad T\'ecnica Federico Santa Mar\'\i a, Casilla 110-V, Valpara\'\i so, Chile.}
\affiliation{Centro Cient\'ifico Tecnol\'ogico de Valpara\'iso, Chile.}

\author{Crist\'obal \surname{Corral}}
\email{cristobal.corral@postgrado.usm.cl}
\affiliation{Departamento de F\'\i sica, Universidad T\'ecnica Federico Santa Mar\'\i a, Casilla 110-V, Valpara\'\i so, Chile.}

\author{Sergey \surname{Kovalenko}}
\email{sergey.kovalenko@usm.cl}  
\affiliation{Departamento de F\'\i sica, Universidad T\'ecnica Federico Santa Mar\'\i a, Casilla 110-V, Valpara\'\i so, Chile.}
\affiliation{Centro Cient\'ifico Tecnol\'ogico de Valpara\'iso, Chile.}

\author{Iv\'an \surname{Schmidt}}
\email{ivan.schmidt@usm.cl}  
\affiliation{Departamento de F\'\i sica, Universidad T\'ecnica Federico Santa Mar\'\i a, Casilla 110-V, Valpara\'\i so, Chile.}
\affiliation{Centro Cient\'ifico Tecnol\'ogico de Valpara\'iso, Chile.}

\begin{abstract}
  We study gravity with torsion in extra dimensions and derive an effective 4-dimensional theory containing four-fermion contact operators at the fundamental scale of quantum gravity in the TeV range. These operators may have an impact on the low-energy observables and can manifest themselves or can be constrained  in precision measurements.  We calculate possible contributions of these operators to some observables at one-loop level. We show that the existing precision data on the lepton decay mode of Z-boson set a stringent limit on the fundamental scale of the gravity to be as high as  $\Lambda\geq \SI{70}{\TeV}$ at 95\% C.L.
\end{abstract}

\maketitle


\section{Introduction}

By now the Standard Model (SM) of particle physics has proved to be a very successful and predictive framework. Recently ATLAS and CMS experiments discovered a new particle with the approximate mass of \SI{125.6}{\GeV}, that is consistent with the longly awaited last missing element of the SM, the Higgs boson~\cite{Aad:2012tfa,Chatrchyan:2012ufa}.
This anticipated triumph will leave, however, many open questions of fundamental kind which do not allow qualifying the SM to be a true fundamental theory but a low energy effective framework. Particularly fundamental of the open questions are a huge hierarchy between the electroweak and gravitational scales as well as the lack of compatibility with the gravity.  There exists in the literature a number/variety of  proposals for the solution of the Hierarchy Problem.  Two of the  most popular of them appealing to the supersymmetry and extra dimensions  are intimately related to the gravity.

There have been many efforts undertaken in the past towards deeper understanding of gravity from different perspectives. As is known the conventional Einstein-Hilbert Theory (EHT)  of gravity can be interpreted as a gauge theory of the Lorentz group~\cite{PhysRev.101.1597}.
On the other hand  classification of particles with definite mass and spin is given in  the flat Minkowski spacetime  in terms of irreducible representations of the Poincar\'e group. Nonetheless, attempts to  construct a gauge theory for the Poincare group in four dimensions have fail~\cite{Kibble:1961ba,PhysRevLett.33.445,PhysRevD.13.3192,PhysRevLett.38.739}.


The EHT might be viewed according to the first order formalism, where the affine connection and the metric are independent variables.  In  cases where the connection  have a non-vanishing antisymmetric part (called \textit{torsion}), this theory is known as the Einstein-Cartan Theory (ECT) of gravity. In the pure gravity case, presence of the torsion does not affect the well known predictions of the EHT. However, coupling fermionic matter to ECT, gives rise to new interactions of the four-fermion type  (see Ref.~\cite{Kibble:1961ba}) absent in the Einstein-Hilbert theory, due to the spin-torsion interaction. Therefore, cosmological or experimental manifestations of these interactions would allow discriminating between these two theories.

However in the four-dimensional spacetime,  the gravity effects in particle interactions at low energies are highly suppressed by the inverse squared of the  Planck mass ($M_{\text{Pl}}\sim\SI{e19}{\GeV}$) making them experimentally unobservable. 

On the other hand the key point of the extra dimensional scenarios  \cite{ArkaniHamed:1998rs,Antoniadis:1998ig,ArkaniHamed:1998nn,Randall:1999ee,Randall:1999vf} for the solution of the Hierarchy Problem is reduction of the fundamental gravity scale down to the \si{\TeV} range.  This implies that the gravity induced  interactions, in particular, those which originates from the torsion become phenomenologically valuable.


The possibility of observation of the torsion-induced interactions has been addressed in the literature~\cite{Belyaev:1998ax,CastilloFelisola:2012fy,Lebedev:2002dp,Kostelecky:2007kx}.


In the present paper we study some phenomenological implications of the torsion induced four-fermion interactions (TFFI) in extra dimensions.
We explore one-loop observables within an effective four dimensional theory derived from the extra dimensional one. We focus on the TFFI contribution to the  $Z$-boson interaction with fermions. Using the existing data on precision tests of the SM~\cite{Altarelli:2004fq,Beringer:1900zz} we extract a stringent limit on the fundamental scale of the gravity. 

The article is organized as follow: In Sec.~\ref{sec:CEF}  we briefly summarize the Cartan-Einstein Formalism. In Sec.~\ref{sec:extradim}  we present an extra dimensional scenario with torsional gravity and derive the corresponding effective four-dimensional theory. Sec.~\ref{sec:oneloop} deals with the calculation of the one-loop $Z$ form factors of the fermions. In Sec.~\ref{sec:constraints} data on precision tests of the SM are used to constraint the parameters of the extra dimensional theory. We conclude with Sec.~\ref{sec:conclusions} summarizing and discussing our results as well as some uncovered aspects of TFFI phenomenology.


\section{\label{sec:CEF} Einstein-Cartan gravity coupled with fermions}


%In this section we give a brief summary of the formalism used in the Einstein-Cartan Theory of gravity. %The ECT is constructed on more general grounds than the Einstein-Hilbert one.
Within the framework of first order formalism, spin connection and vielbeins are independent fields, and  torsion might not vanish. Including torsion implies the existence of an antisymmetric part of the affine connection
\begin{align}
  \hat{\mathcal{T}}_{\hat{\mu}}{}^{\hat{\lambda}}{}_{\hat{\nu}} \equiv 2\hat{\Gamma}_{[\hat{\mu}}{}^{\hat{\lambda}}{}_{\hat{\nu}]} = \hat{\Gamma}_{\hat{\mu}}{}^{\hat{\lambda}}{}_{\hat{\nu}} - \hat{\Gamma}_{\hat{\nu}}{}^{\hat{\lambda}}{}_{\hat{\mu}},
\end{align}
where hatted  indices denote coordinates on a $D$-dimensional  spacetime, $\Ma$, endowed with a metric $\hat{g}_{\hat{\mu}\hat{\nu}}(x)$,  related with the vielbeins, $\vih^{\hat{a}}_{\hat{\mu}}(x)$, via
\begin{equation}
  \label{metricrelation}
  \hat{g}_{\hat{\mu}\hat{\nu}}(x) = {\eta}_{\hat{a}\hat{b}}\,\vih^{\hat{a}}_{\hat{\mu}}(x)\,\vih^{\hat{b}}_{\hat{\nu}}(x),
\end{equation}
and $\etah{a}{b} = \diag{\left(-,+,\ldots,+\right)}$ is the $D$-dimensional Minkowski metric on the tangent space, $T_x\Ma$.

In  differential forms,  torsion and curvature are defined in terms of the vielbeins and spin connection through the Cartan structure equations,
\begin{align}
  \label{cartantorsion}
  \df\vifhn{a}{} + \wfh^{\hat{a}}{}_{\hat{c}}\wedge\vifhn{c}{} &= \Tfhn{a}{} \equiv \frac{1}{2!}\hat{\T}_{\hat{\mu}}{}^{\hat{a}}{}_{\hat{\nu}}\,dx^{\hat{\mu}}\wedge dx^{\hat{\nu}}, \\
  \label{cartancurvature}
  \df\wfh^{\hat{a}\hat{b}} + \wfh^{\hat{a}}{}_{\hat{c}}\wedge\wfh^{\hat{c}\hat{b}} &= \Rfhn{a}{}{b}{} \equiv \frac{1}{2!}\Rh^{\hat{a}\hat{b}}{}_{\hat{\mu}\hat{\nu}}\,dx^{\hat{\mu}}\wedge dx^{\hat{\nu}},
\end{align}
where $\vifhn{a}{}$ and $\wfh^{\hat{a}}{}_{\hat{c}}$ are the vielbein and spin connection 1-forms, while $\Tfhn{a}{}$ and $\Rfhn{a}{}{b}{}$ are the  torsion and curvature 2-forms. Therefore, the latter serves to write down the ECT of gravity.

The fermionic matter is introduced via the minimal coupling procedure, by  defining the covariant derivative for fermions
\begin{equation}
  \label{covder}
  D_{\hat{a}}\Psi = \hat{E}_{\hat{a}}^{\hat{\mu}}D_{\hat{\mu}}\Psi = \hat{E}_{\hat{a}}^{\hat{\mu}}\left(\partial_{\hat{\mu}}\Psi + \frac{1}{4}(\hat{\omega}_{\hat{\mu}}){}^{\hat{b}\hat{c}}\gamma_{\hat{b}\hat{c}}\Psi\right),
\end{equation}
that keeps invariant the Dirac action under local Lorentz transformation. Above, we used the inverse vielbein, $\hat{E}^\mu_a = \left(\hat{e}^a_\mu\right)^{-1}$, and in general $\gamma_{a_1 \cdots a_n} = \gamma_{[a_1}\cdots\gamma_{a_n]}$.
%% The spin connection can be usefully decomposed into 
%% \begin{align}
%%   \hat{\omega}_{\hat{\mu}}{}^{\hat{a}\hat{b}} = \hat{\free{\omega}}_{\hat{\mu}}{}^{\hat{a}\hat{b}} + \hat{\K}_{\hat{\mu}}{}^{\hat{a}\hat{b}}
%% \end{align}
%% where $\hat{\free{\omega}}_{\hat{\mu}}{}^{\hat{a}\hat{b}}$ is the torsion-free spin-connection, intimately related with the Levi-Civita connection and  
%% \begin{equation}
%%   \label{generalcontorsion}
%%   \hat{\K}_{\hat{\mu}}{}^{\hat{b}}{}_{\hat{c}} = \frac{1}{2}\left(\hat{\T}_{\hat{\mu}}{}^{\hat{b}}{}_{\hat{c}} - \hat{\T}_{\hat{\mu}\hat{c}}{}^{\hat{b}} + \hat{\T}^{\hat{b}}{}_{\hat{\mu}\hat{c}}\right).
%% \end{equation}
%% is the contorsion tensor which encondes the torsion information in the spin-connection. Hereon, circled quantities are torsion-free and boldface symbols denote differential forms. The two most important equations in this formalism, that relate the vielbeins and the spin connection with torsion and curvature respectively are the so called {\it{Cartan structure equations}} 
%% \begin{align}
%%   \label{cartantorsion}
%%   \df\vifhn{a}{} + \wfh^{\hat{a}}{}_{\hat{c}}\wedge\vifhn{c}{} &= \Tfhn{a}{} \equiv \frac{1}{2!}\hat{\T}_{\hat{\mu}}{}^{\hat{a}}{}_{\hat{\nu}}\,dx^{\hat{\mu}}\wedge dx^{\hat{\nu}}, \\
%%   \label{cartancurvature}
%%   \df\wfh^{\hat{a}\hat{b}} + \wfh^{\hat{a}}{}_{\hat{c}}\wedge\wfh^{\hat{c}\hat{b}} &= \Rfhn{a}{}{b}{} \equiv \frac{1}{2!}\Rh^{\hat{a}\hat{b}}{}_{\hat{\mu}\hat{\nu}}\,dx^{\hat{\mu}}\wedge dx^{\hat{\nu}},
%% \end{align}
%% where $\vifhn{a}{}$ is the $1$-form vielbein, $\Tfhn{a}{}$ and $\Rfhn{a}{}{b}{}$ are the $2$-forms torsion and curvature respectively.

%% \subsection{Action and equations of motion}
%% %The following action will be considered

In the context of the ECT, we are considering the following action (bold symbols represent differential forms)
\begin{align}
  \nonumber
  S &= \frac{1}{2\kappa_*^2}\int\frac{\epsilon_{\hat{a}_1\ldots\hat{a}_D}}{(D-2)!}\,\Rfh^{\hat{a}_1\hat{a}_2}\wedge\vifh^{\hat{a}_3}\wedge\ldots\wedge\vifh^{\hat{a}_D} \\
  \nonumber
  &\quad- \sum_{r}\int\bigg(\frac{1}{2}\left(\bar{\Psi}_r\gf\wedge\star\Df\Psi_r - \Df\bar{\Psi}_r\wedge\star\gf\Psi_r\right) \\
  \label{formaction}
  &\qquad + m_r \bar{\Psi}_r\Psi_r\,\frac{\epsilon_{\hat{a}_1\ldots\hat{a}_D}}{D!}\vifh^{\hat{a}_1}\wedge\ldots\wedge\vifh^{\hat{a}_D}\bigg)
\end{align}
where $\kappa_*^2 = 8\pi G_* = M_*^{-(2+n)}$, with $G_*$ and $M_*$  the analog of the Newtonian gravity constant and the reduced Planck mass in $D$ dimensions. In the fermionic sector, $\bar{\Psi}\equiv\imath\Psi^\dagger\gamma^0$ is the Dirac adjoint,  $r$ index indicates flavor, \mbox{$\gf=\gamma_{\hat{a}}{\bf{e}}^{\hat{a}}$} is the gamma matrix 1-form, \mbox{$\Df=D_{\hat{a}}{\bf{e}}^{\hat{a}}$}, and the symbol $\hodge$ denotes Hodge duality.

The equations of motion are found from the principle of least action, and yield
\begin{align}
  \label{einsteineom}
  \Rh_{\hat{a}\hat{b}} - \frac{1}{2}\etah{a}{b}\Rh &= \kappa_*^2\,\hat{T}_{\hat{a}\hat{b}} \\
  \label{contorsionfound}
  \Th_{\hat{a}}{}^{\hat{b}}{}_{\hat{c}} = 2\,\hat{\K}_{\hat{a}}{}^{\hat{b}}{}_{\hat{c}} &= -\, \frac{\kappa_*^2}{2}\sum_{r}\bar{\Psi}_r\gamma_{\hat{a}}{}^{\hat{b}}{}_{\hat{c}}\Psi_r.
\end{align}
where $\hat{T}_{\hat{a}\hat{b}}$ is the energy-momentum tensor of fermions, and $\hat{\K}_{\hat{a}}{}^{\hat{b}}{}_{\hat{c}}$ is the contorsion tensor. From Eq.~\eqref{cartantorsion}, the spin connection can be split into a torsion-free part plus the contorsion,
\begin{align}
  \hat{\omega}_{\hat{\mu}}{}^{\hat{a}\hat{b}} &= \hat{\free{\omega}}_{\hat{\mu}}{}^{\hat{a}\hat{b}} + \hat{\K}_{\hat{\mu}}{}^{\hat{a}\hat{b}}\\
  \intertext{and additionally}
  \Tfhn{a}{} &= {\hat{{\boldsymbol{\mathcal{K}}}}^{\hat{a}}{}_{\hat{b}}}\wedge \hat{{\bf{e}}}^{\hat{b}}  .
\end{align}

%% For the fermionic action (\ref{formaction})
%% %the energy-momentum tensor is
%% it takes the form
%% \begin{align}
%%   \hat{T}_{\hat{a}\hat{b}} &= \sum_{r}\frac{1}{2}\left(\bar{\Psi}_r\gamma_{\hat{a}}D_{\hat{b}}\Psi_r - D_{\hat{b}}\bar{\Psi}_r\,\gamma_{\hat{a}}\Psi_r\right) + \etah{a}{b}\Lag_{\Psi}
%% \end{align}
%% with $\Lag_\Psi$ being the Dirac Lagrangian derived from Eq.~\eqref{formaction}. The second equation of motion can be obtained varying with respect to the torsionful spin connection. Solving the algebraic equation one find a completely antisymmetric torsion tensor for this theory. Noting that contorsion tensor in Eq.~\eqref{generalcontorsion} is antisymmetric in its last two indices, the equation of motion for the spin connection reads
%% \begin{align}
%%   \label{contorsionfound}
%%   \hat{\K}_{\hat{a}}{}^{\hat{b}}{}_{\hat{c}} = \frac{1}{2}\,\Th_{\hat{a}}{}^{\hat{b}}{}_{\hat{c}} = -\, \frac{\kappa_*^2}{4}\sum_{r}\bar{\Psi}_r\gamma_{\hat{a}}{}^{\hat{b}}{}_{\hat{c}}\Psi_r.
%% \end{align}
%% One could notice that gravity and fermions in CEF leads to a completely antisymmetric torsion tensor. 
% Using Eq.~\eqref{generalcontorsion} the contorsion tensor is  found to be
% \begin{align}
%   \label{contorsionfound}
%   \hat{\K}_{\hat{a}}{}^{\hat{b}}{}_{\hat{c}} &= - \frac{\kappa_*^2}{4}\sum_{r}\bar{\Psi}_r\gamma_{\hat{a}}{}^{\hat{b}}{}_{\hat{c}}\Psi_r.
% \end{align}

%% \subsection{Torsional contribution to the fermionic action}

%% Eq.~\eqref{contorsionfound} has no dynamics, therefore one can sustitute into the initial action as a constraint. There exists models where torsion appears as a dynamical field instead of the minimal consideration of this work (for further reading Ref.~\cite{Carroll:1994dq,Belyaev:1998ax}). Replacing Eq.~\eqref{contorsionfound} in gravity sector leads to

The equation of motion for the spin connection (Eq.~\eqref{contorsionfound}) is algebraic, therefore it can be substituted into the initial action in order to eliminate the torsion, which acts in this model as an auxiliary field. Towards this end it is convenient to pinpoint the torsion in the terms of Eq.~(\ref{formaction}) by the following decompositions
\begin{eqnarray}
  \label{gravdecomp}
  \Rfhn{a}{}{b}{} &=& \Rfhnfree{a}{}{b}{} + \free{\Df}\Kfhn{a}{}{b}{} + \Kfhnud{a}{}{c}{}\wedge\Kfhn{c}{}{b}{},\\
  \label{F-decomp}
  \Df\Psi_r \ \, &=& \free{\Df}\Psi_r + \frac{1}{4}\Kfhn{a}{}{b}{}\,\gamma_{\hat{a}\hat{b}}\Psi_r,
\end{eqnarray}
where as before the circled quantities are torsion-free. Then integrating out the torsion in Eq. \eqref{formaction} one finds the action
\begin{align}\label{4FI}
  S &= \free{S}_\text{grav} + \free{S}_\Psi + \frac{\kappa_* ^2}{32}\sum_{r,s}\int \dv[D]\,(\bar{\Psi}_r\gamma^{\hat{a}\hat{b}\hat{c}}\Psi_r)\,(\bar{\Psi}_s\gamma_{\hat{a}\hat{b}\hat{c}}\Psi_s)
\end{align}
with the contact four fermion interactions. Their presence is a peculiar prediction to the ECT. The following two features of the torsion-induced four fermion interactions should be highlighted. First, they conserve lepton flavors due to flavor blindness of gravity. Second, fermions in these interactions are flavor paired due to the fact that the torsion-fermion interactions arise from the kinetic term of  the Dirac action. 

The next comment is also in order. In the torsional extensions of the gravity the measure is not unique.
We consider the minimal version of the ECT with the standard measure, although more general theories can be constructed if quadratic terms in the torsion are considered, see Ref.~\cite{PhysRevD.18.2730,PhysRevD.22.1915,Baekler2011,Fabbri:2011kq}. Within this framework, gauge fields do not couple to the torsion at classical level. 
This is because the one-form gauge field are singlets of local Lorentz transformations and no covariant derivative is needed (for more details see Ref.~\cite{Benn:1980ea}). Gauge invariance of the theory in this case is also guaranteed~\cite{Hehl:1976kj}. 
Demanding gauge invariance for fermions, the covariant derivative from Eq.~\eqref{covder} will be shifted by gauge connections $\hat{V}_{\hat{\mu}}^A$, in order to keep the action invariant under such transformation and interaction between fermions and gauge bosons will arise.


\section{\label{sec:extradim}Extra dimensions scenario}


%\subsection{Motivations}

A distinctive feature of the ECT is the presence of the four-fermion contact operators discussed in the previous section.  Their manifestation in particle interactions  could be a smoking gun of the spacetime torsion. However as seen from Eq.~\eqref{4FI} these operators are  suppressed  by the inverse of the squared Planck mass leaving them experimentally unavailable.

However, the situation may dramatically changes in extra dimensions where the fundamental Planck scale can be reduced down to the \si{\TeV} range. The corresponding extra dimensional scenarios have been proposed for solution of the Hierarchy Problem. The most popular of them are the scenarios  with more than two compact extra dimensions, proposed by Arkani-Hamed, Dimopoulos and Dvali~\cite{ArkaniHamed:1998rs,Antoniadis:1998ig,ArkaniHamed:1998nn}, and with only one but large extra dimension of Randall and Sundrum~\cite{Randall:1999ee,Randall:1999vf}. A few generalizations of these scenarios have been considered in Ref.~\cite{DeWolfe:1999cp,Gremm:1999pj,MPS,CastilloFelisola:2004eg}.


We consider the ECT in the five-dimensional spacetime with the Randall-Sundrum metric~\cite{Randall:1999ee}
\begin{align}
  \label{RSmetric}
  ds^2 = e^{-2k\abs{y}}\eta_{\mu\nu}dx^\mu\,dx^\nu + dy^2,
\end{align}
where the fifth dimension, $y$, is compactified on an  $S^1/\mathbb{Z}_2$ orbifold, corresponding to the interval \mbox{$0\leq y\leq \pi R$}. 


The Kaluza-Klein (KK) decomposition of five-dimensional fermions into chiral four-dimensional ones~\cite{Kehagias:2000au,MPT,CastilloFelisola:2010xh,CastilloFelisola:2012ez}, taking into account only the zero KK modes for phenomenological reasons and using the profiles from Ref.~\cite{Gherghetta:2000qt,Gherghetta:2006ha} is
\begin{align}
  \Psi_r(x,y) &= f(c_r,y)\bigg(\psi_+(x) + \psi_-(x)\bigg)\\
  \intertext{with}
  f(c_r,y) &= \sqrt{\frac{k\left(1-2c_r\right)}{e^{(1-2c_r)\pi\,kR}-1}}\,e^{(2-c_r)ky},
\end{align}
where we have chosen the same profile for left and right handed for simplicity. The $c_i$'s coefficients control the localization of fermions. For $c_i>\tfrac{1}{2}$ and $c_i<\tfrac{1}{2}$ they are localized near the Planck and the \si{\TeV} branes respectively, while for $c=\tfrac{1}{2}$ fermions lie in the bulk.

The Clifford algebra in five dimensions can be constructed using the four-dimensional one. The gamma matrices in five-dimensional spacetime are
\begin{align}
  \ghu{a} = \left(\gamma^a,\gamma^*\right).
\end{align}
With this definition the product of gamma matrices in Eq.~\eqref{4FI} is 
\begin{align}
  (\ghhhu{a}{b}{c})(\ghhhd{a}{b}{c}) &= 6\left(\gamma_a\gamma^*\right)\left(\gamma^a\gamma^*\right) + 3\left(\gamma^{ab}\gamma^*\right)\left(\gamma_{ab}\gamma^*\right).
\end{align}


Using the chirality condition $\gamma^*\psi_{r\pm} = \pm\psi_{r\pm}$, the torsion induced four fermion interactions in Eq.~\eqref{4FI}, can be written in the zero mode approximation as (see Ref.~\cite{Castillo-Felisola:2013jva})
\begin{widetext}
  \begin{align}
    \nonumber
    S_{4\text{FI}} &\approx \sum_{r,s}\,\frac{\kappa_{\text{eff}}^2}{32}\,\int d^4x\bigg\{6\bigg[\left(\bar{\psi}_{r+}\gamma^\mu\psi_{r+}\right)\left(\bar{\psi}_{s+}\gamma_\mu\psi_{s+}\right) + \left(\bar{\psi}_{r-}\gamma^\mu\psi_{r-}\right)\left(\bar{\psi}_{s-}\gamma_\mu\psi_{s-}\right) - \left(\bar{\psi}_{r+}\gamma^\mu\psi_{r+}\right)\left(\bar{\psi}_{s-}\gamma_\mu\psi_{s-}\right) \\ 
      \label{4FI5D}
      &\qquad - \left(\bar{\psi}_{r-}\gamma^\mu\psi_{r-}\right)\left(\bar{\psi}_{s+}\gamma_\mu\psi_{s+}\right)\bigg] +  3\bigg[\left(\bar{\psi}_{r+}\gamma^{\mu\nu}\psi_{r-}\right)\left(\bar{\psi}_{s+}\gamma_{\mu\nu}\psi_{s-}\right) + \left(\bar{\psi}_{r-}\gamma^{\mu\nu}\psi_{r+}\right)\left(\bar{\psi}_{s-}\gamma_{\mu\nu}\psi_{s+}\right)\bigg]\bigg\},
  \end{align}
where 
\begin{align}
  \label{kapparel}
  k_{\text{eff}}^2 \equiv \frac{(2c_r -1)(2c_s - 1)\left(e^{-2\pi\,kR\left(c_r + c_s -1\right)} - 1\right)\,\kappa_*^2\,k}{(4 - 2c_r - 2c_s)\left(e^{\pi\,kR\left(1 - 2c_r\right)}-1\right)\left(e^{\pi\,kR\left(1 - 2c_s\right)}-1\right)}.
\end{align}
\end{widetext}

Note that the axial-tensor term in Eq.~\eqref{4FI5D} must be discarded by phenomenological reasons.  This is required by the presence of chiral fermions in the four-dimensional effective theory leading, as demonstrated in Ref.~\cite{Flachi:2001bj}, to the orbifold boundary condition $\pm\gamma^*f_r(y)=f_r(-y)$.  Since before the dimensional reduction the term $\bar{\Psi}_r\gamma^{\mu\nu}\gamma^*\Psi_r$ is odd under $y\rightarrow-y$, therefore it must vanish identically~\cite{Lebedev:2002dp}. Thus, we are left with only axial-vector torsion-induced interaction in Eq.~\eqref{4FI5D}.



%If one wants chiral fermions in the effective theory in $4$-dimensions the orbifold boundary condition $\pm\gamma^*f_r(y)=f_r(-y)$ must be satisfied, as Ref.~%
%\cite{Flachi:2001bj} shown. Analysing the last term on Eq.~\eqref{4FI5D} one can verify that, before the dimensional reduction, the term $\bar{\Psi}_r\gamma^{\mu
%\nu}\gamma^*\Psi_r$ is odd under $y\rightarrow-y$. This leads to a trivial contribution of axial-tensor current interaction and torsion four fermion interaction is %decomposed into purely axial-vector one (see Ref.~\cite{Lebedev:2002dp}).

Using the definitions $\kappa_{\text{eff}}^2 \equiv M_{\text{Pl}}^{-2}$ and $\kappa_*^2 \equiv M_*^{-3}$ and the stabilization value $kR\sim10$ (see Ref.~\cite{Goldberger:1999uk}), we obtain %for different values of $c_i$'s
%differents values $c_i$
%One can consider the stabilization value $kR\sim10$ from Ref.~\cite{Goldberger:1999uk} and differents values $c_i$
\begin{align}
  M_{\text{Pl}}^2 \approx 
  \begin{cases}    
    \num{5e-27} \dfrac{M_*^3}{k};  & c_r\simeq c_s\simeq 0 \\[2ex]
    \num{e-24} \dfrac{M_*^3}{k};  & c_r\simeq c_s\simeq 1/2 \\[2ex]
    \num{e-2}\dfrac{M_*^3}{k};  & c_r\simeq c_s\simeq 1 
  \end{cases}
\end{align}

% In the low-energy four dimensional effective theory the gauge fields $V^{A}_{\mu}$ in Eq. \eqref{GF-4dim} are photon and $Z$-boson with the couplings 
% $g_{\text{eff}} = e\,Q_f$  and $g_{\text{eff}} = e/(2c_Ws_W)\left(T^3_f - 2\,s_W^2\,Q_f\right)$ respectively. Here $Q_f$ and  $T_f^3$ are the electric charge and the third component of the weak isospin. We also denoted  $s_W(c_W) = \sin(\cos)\theta_W$. 


%In order to analyse only the fundamental scale of extra dimensions, the effective coupling of neutral gauge bosons coupled to fermionic current, will be considered %equal as the SM in $4D$, that is, $g_{\text{eff}} = e\,Q_f$ for photon exchange and $g_{\text{eff}} = e/(2c_Ws_W)\left(T^3_f - 2\,s_W^2\,Q_f\right)$ for $Z^0$ %exchange, where $Q_f$ is electric charge in proton unities, $T_f^3$ is the third component of weak isospin and $s_W(c_W) = \sin(\cos)\theta_W$. 


%\section{One loop calculations and form factors}

In this section, the effects of curvature in the effective theory in $4$-dimensions will be ignored, by the fact that the Universe is essentially flat as Ref.~\cite{Larson:2010gs} indicates. This assumption is also based in our comparision of torsion effects with particle accelerators data (the predominant forces in these experiments becames from the SM interactions, and the curvature effects are negligible). Obviously, this assumption is not valid anymore where curvature effects can not be droped, i.e.: near to a black hole or neutron star. This consideration has been used before in Ref.~\cite{Carroll:1994dq,Belyaev:1998ax}. In CEF the metric and connection are independent, this condition allows to curvature and torsion be independent too. There exist manifolds with torsion and no curvature, where teleparallel gravity relies (for further reading, see Ref.~\cite{Arcos:2005ec}). This special kind of manifolds are called Weitzenb\"ock manifolds.

Our interest by now, is try to extract information of the contribution of torsion to one loop form factors. Now, considering that $SU(2)_L\otimes U(1)_Y$ gauge sector of the SM is torsion free, the only effect of torsion is through the four fermion contact term in Eq.~\eqref{4FI5D}. Using this kind of interaction, our aim is to do one loop calculation in this theory. 

The process to be calculated in this section can be extracted from the general four femionic interaction Lagrangian used in Ref.~\cite{GonzalezGarcia:1998ay} plus one neutral gauge boson exchange. The relevant Lagrangians used by Gonzalez-Garc\'ia, Gusso and Novaes in the previous reference are
 \begin{align}
  \nonumber
   \Lag_{\text{V}} &= \eta_V\,\frac{g^2}{\Lambda^2}\left[\psi_r\gamma_\mu\left(V_V - A_V\gamma_5\right)\psi_r\right] \\ 
  \label{lagvec}
  &\espacio\espacio\espacio\times \left[\psi_s\gamma^\mu\left(V_V - A_V\gamma_5\right)\psi_s\right], \\
  \nonumber
    \Lag_{\text{T}} &= \eta_T\,\frac{g^2}{\Lambda^2}\left[\psi_r\sigma_{\mu\nu}\left(V_T - A_T\gamma_5\right)\psi_r\right]\\ 
     \label{lagten}
     &\espacio\espacio\espacio\times\left[\psi_s\sigma^{\mu\nu}\left(V_T - A_T\gamma_5\right)\psi_s\right],
 \end{align}
where $r$ and $s$ denotes flavor indices as in the previous sections and in the following, the notation of Ref.~\cite{GonzalezGarcia:1998ay} for evaluating form factors will be used. The most general one loop calculation in this theory, can be build with the previous four fermion interaction Lagrangian and gauge boson coupled to fermions
\begin{align}
\label{feyndiagram}
  \begin{tikzpicture}[thick,baseline=(current  bounding  box.center)]
    \coordinate (V) at (0,0);
    \node[circle,draw=black,shade,minimum size=.6cm]  at (V)  {};
    \draw[boson] (-2,0) node[anchor=south] {$V_\mu(k)$} -- (180:3mm);
    \draw[directed] (1,-1) node[anchor=west] {$f(p)$}  -- (-45:3mm);
    \draw[directed] (45:3mm) -- (1,1) node[anchor=west] {$f(p')$};
  \end{tikzpicture}
  =\imath e \, V_\mu(k) J^\mu(p,p')
\end{align}
%% \begin{figure}[H]
%% \scalebox{1.5}[1.5]{\begin{tikzpicture}
%%  \draw[boson,thick] (-5,0) -- (-3.3,0);
%%  \node[above,scale=.7] at (-4,.2) {$Z^0_\mu(k),A_\mu(k)$};
%%  \draw[thick,directed] (-2,-1) -- (-3,0);
%%  \node[right,scale=.7] at (-2.3,-.5) {$f(p)$};
%%  \draw[thick,directed] (-3,0) -- (-2,1);
%%  \node[right,scale=.7] at (-2.3,.5) {$f(p')$};
%%  \filldraw[shade] (-3,0) circle (.3);
%%  \node[scale=.6] at (-2,0) {$=$};
%%  \node[scale=.6] at (-.8,0) {$(i\,e)V_\mu(k) J^\mu(p,p')$};
%% \end{tikzpicture}}
%% \caption{Feynman diagram for one loop process including gauge coupling, four fermion interaction due torsion and general neutral current $J^\mu(p,p')$.}
%% \end{figure}
where $V_\mu(k) = \{A_\mu(k),Z^0_\mu(k)\}$ are the neutral gauge bosons to be considered and
\begin{align}
  J^\mu(p,p') &\equiv \bar{u}(p')\Bigg[\gamma^\mu\,F_V(k^2) +F_A(k^2)\gamma^\mu\gamma_5  \label{current} \\
    \nonumber
    &\quad + i\frac{\sigma^{\mu\nu}\,k_\nu}{2\,m_f}F_M(k^2) + F_D(k^2)\frac{1}{2m_f}\sigma^{\mu\nu}\gamma_5 k_\nu\Bigg]u(p)
\end{align}
is the more general neutral current constructed from Eq.~\eqref{lagvec} and \eqref{lagten}. $F_i(k)$, where $i=V,A,M,D$ denotes vector, axial, magnetic and dipole form factor respectively that plays an important role in precise measurments of radiative correction. In order to reproduce low energy regions with two possible neutral gauge boson exchange, the following condition for form factors must be satisfied. If one consider photon coupled to $J^\mu(p,p')$ in Eq.~\eqref{current}
\begin{align}
F_V^\gamma(0) &= Q_f,  \\
F_A^\gamma(0) &= 0, \\
F_M^\gamma(0) &= a_f^\gamma \equiv \frac{1}{2}\left(g_f-2\right),\\
F_D^\gamma(0) &= d_f^e\,\frac{2\,m_f}{e},
\end{align}
must be met, where $Q_f$, $a_f^\gamma$ and $d_f^e$ denotes unities of proton electric charge, the anomalous magnetic moment and electric dipole moment of the fermion $f$ respectively. Considering $Z^0$ coupled to $J^\mu(p,p')$ in Eq.~\eqref{current}
\begin{align}
F_V^{Z^0}(0) &= \frac{1}{2\,s_W\,c_W}\left(T_3^f - 2\,Q_f\,s_W^2\right), \\
F_A^{Z^0}(0) &=  \frac{1}{2\,s_W\,c_W}T_3^f, \\
F_M^{Z^0}(0) &= a_f^Z, \\
F_D^{Z^0}(0) &= d_f^w\,\frac{2\,m_f}{e}, 
\end{align}
must also satisfied, where $s_W(c_W) = \sin(\cos)\theta_W$ and $T_3^f$, $a_f^Z$ and $d_f^w$ denotes the third component of weak isospin, fermion weak magnetic moment and weak dipole moment of the fermion $f$ respectively. Comparing the general four fermionic interaction Lagrangians in Eq.~\eqref{lagvec} and \eqref{lagten} with Eq.~\eqref{4FI5D} coming from $D = 5$ torsionful manifold, one can identify
\begin{align}
\label{vectpar}
  V_V = 0 \espacio ; \espacio &A_V = 1 \espacio; \espacio \eta_V = +6 \\
\label{tenpar}
  V_T = 0 \espacio ; \espacio &A_T = 1 \espacio; \espacio \eta_T = +3
 \end{align}
The normalizations $g^2/4\pi = 1$ (if one consider different flavors on the loop i.e.: t-channel) and $g^2/2\pi = 1$ (if one consider the same flavors on the loop i.e.: s-channel) has been used. With the previous consideration, one could decompose the forms factors in their tree level value plus a contribution due radiative correction at one loop
\begin{align}
F_i^B(k^2) = F_i^{B\,\text{tree}} + \delta F_i^B(k^2),
\end{align}
where $i=V,A,M,D$ and $B=\gamma,Z^0$. Calculating this radiative corrections form factors $\delta F_i^B(k^2)$ is straighforward using the results obtained in Ref.~\cite{GonzalezGarcia:1998ay}. In the present work, $s$ and $t$-channels has been used with electrons in the final state and considering all possible particles running into the loop, giving for photon coupling
\begin{align}
 \delta F_V^\gamma(k^2) &= \frac{6}{\pi}\,\frac{k^2}{\Lambda^2}\,\ln\left(\frac{\Lambda^2}{\mu^2}\right), \\
 \delta F_A^\gamma(k^2) &= 0 , \\
 \delta F_M^\gamma(k^2) &= 0.220411\,\left(\frac{[\text{GeV}]}{\Lambda}\right)^2\,\ln\left(\frac{\Lambda^2}{\mu^2}\right), \\
 \delta F_D^\gamma(k^2) &= 0.
\end{align}
The same considerations has been used for $Z^0$ boson coupling with electrons in the final state. The following results has been obtained
\begin{align}
 \delta F_V^{Z^0}(k^2) &= -0.18084\,\frac{k^2}{\Lambda^2}\,\ln\left(\frac{\Lambda^2}{\mu^2}\right), \\
 \nonumber
 \delta F_A^{Z^0}(k^2) &= \left[9.56024\times10^{-2}\,\frac{k^2}{\Lambda^2} \right. \\
 & \left.+6.869\times10^4\,\left(\frac{[\text{GeV}]}{\Lambda}\right)^2\,\right]\,\ln\left(\frac{\Lambda^2}{\mu^2}\right), \\
 \delta F_M^{Z^0}(k^2) &= 7.9147\times10^{-2}\,\left(\frac{[\text{GeV}]}{\Lambda}\right)^2\,\ln\left(\frac{\Lambda^2}{\mu^2}\right), \\
 \delta F_D^{Z^0}(k^2) &= 0,
\end{align}
where $\mu$ denotes the scale involved in the process. If one consider all fermions in the loop, one could choose the mass of the heavier running fermion as the major mass scale involved in the process. In the following $\mu=m_t$ will be used.

An interesting thing happened by considering extra dimension scenario. If one focus only in $4$ dimensional scenario, the axial-tensor term in \eqref{4FI5D} disappear and the parameters on Eq.~\eqref{tenpar} vanishes, leading to no contribution to $\delta F_M^\gamma(k^2)$ and $\delta F_M^{Z^0}(k^2)$ in this theory. The appearance of magnetic forms factors in this theory is a purely effect of the fifth dimension. 



\section{\label{sec:oneloop}One-loop calculations and form factors}


In the following, the effects of curvature in the effective theory in four dimensions will be ignored, by the fact that the Universe is essentially flat as Ref.~\cite{Larson:2010gs} indicates. This consideration has been used before in Ref.~\cite{Carroll:1994dq,Belyaev:1998ax,Kostelecky:2007kx}, and allows us to discriminate between the EHT and the ECT of gravity. Moreover,  this  condition is compatible with the independence of Riemaniann curvature and torsion. 

Our interest by now, is to extract information of the contribution of torsion to one-loop form factors. Considering that $SU(2)_L\otimes U(1)_Y$ gauge sector of the SM is torsion free, the only effect of torsion is through the four-fermion contact term in Eq.~\eqref{4FI5D}. Using this kind of interaction, our aim is to do one-loop calculation in this theory.

The process to be calculated in this section can be extracted from the general four-femion interaction Lagrangian used in Ref.~\cite{GonzalezGarcia:1998ay} plus one neutral gauge boson exchange. The relevant Lagrangian used by Gonzalez-Garc\'ia, Gusso and Novaes in the previous reference is
\begin{align}
  \nonumber
  \Lag_{\text{V}} &= \eta_V\,\frac{g^2}{\Lambda^2}\left[\psi_r\gamma_\mu\left(V_V - A_V\gamma^*\right)\psi_r\right] \\ 
  \label{lagvec}
  &\espacio\espacio\espacio\times \left[\psi_s\gamma^\mu\left(V_V - A_V\gamma^*\right)\psi_s\right],
\end{align}
where $r$ and $s$ denotes flavor indices as in the previous sections. The most general one-loop calculation in this theory, can be build with the previous four-fermion interaction Lagrangian and gauge boson coupled to fermions
\begin{align}
  \label{feyndiagram}
  \begin{tikzpicture}[thick,baseline=(current  bounding  box.center)]
    \coordinate (V) at (0,0);
    \node[circle,draw=black,shade,minimum size=.6cm]  at (V)  {};
    \draw[boson] (-2,0) node[anchor=south] {$V_\mu(k)$} -- (180:3mm);
    \draw[directed] (1,-1) node[anchor=west] {$f(p)$}  -- (-45:3mm);
    \draw[directed] (45:3mm) -- (1,1) node[anchor=west] {$f(p')$};
  \end{tikzpicture}
  =\imath e \, V_\mu(k) J^\mu(p,p')
\end{align}
where $V_\mu = \{\gamma_\mu, Z^0_\mu\}$ are the neutral gauge bosons to be considered, and
\begin{widetext}
  \begin{align}
    \label{current}
    J^\mu(p,p') &\equiv \bar{u}(p')\Bigg[\gamma^\mu\,F_V(k^2) +F_A(k^2)\gamma^\mu\gamma^* + i\frac{\sigma^{\mu\nu}\,k_\nu}{2\,m_f}F_M(k^2) + F_D(k^2)\frac{1}{2m_f}\sigma^{\mu\nu}\gamma^* k_\nu\Bigg]v(p)
  \end{align}
\end{widetext}
is the more general neutral current constructed from Eq.~\eqref{lagvec}. The $\{F_i(k)\}$ denote vector, axial, magnetic and dipole form factors, that play an important role in precise measurments of radiative correction. 

We decompose form factors in their tree level value plus a contribution due radiative correction at one-loop $F_i(k^2) = F_i^{\text{tree}} + \delta F_i(k^2)$. Calculating this radiative corrections form factors $\delta F_i(k^2)$ is straighforward using the results  in Ref.~\cite{GonzalezGarcia:1998ay}.

In the present work, $s$ and $t$-channels has been used with electrons in the final state and considering all possible particles running into the loop. By considering quarks into the loop, a factor $3$ has to be taken into account due their color. The normalization $g^2/4\pi = 1$ has been used.

If $J^\mu(p,p')$ in Eq.~\eqref{current} is coupled to a photon field, the only non-vanishing form factor is $F_V^\gamma$, due the absence of axial-tensor interaction from the orbifold condition in RS scenario. This leads to no contribution to fermionic anomalous magnetic moment.

Considering $Z^0$ boson coupled to $J^\mu(p,p')$ and electrons in the final state ($\mu = M_Z$ as the scale of the process involving $Z^0$ boson), the non vanishing form factors, evaluated at $Z^0$ pole are  
\begin{align}
  \delta F_V^Z(M_Z^2) &= \num{28.7} \, \left(\frac{\si{\GeV}^2}{\Lambda^2}\right)\,\ln\left(\frac{\Lambda^2}{M_Z^2}\right), \\
  \delta F_A^Z(M_Z^2) &= -\num{3.43e4} \, \left(\frac{\si{\GeV}^2}{\Lambda^2}\right)\,\ln\left(\frac{\Lambda^2}{M_Z^2}\right).
\end{align}


\section{\label{sec:constraints}Constraints from precision tests: Z boson width decay}

%Here, our aim is to compare our theoretical results with experimental data.
The obtained form factors can be used to strongly constraint the scale for new physics coming from extra dimensions scenarios. This strength comes from comparing one-loop observables with precision tests of the SM. One of the most well known values, with excellent statistic and precision is the $Z^0$ width decay. Considering the most general current $J^\mu(p,p')$ in Eq.~\eqref{current}, the width decay of $Z^0$ into electrons can be decomposed
\begin{align}
  \Gamma_{\text{th}}\left(Z^0\rightarrow e^+\,e^-\right) &= \Gamma_{\text{SM}} + \delta\Gamma_{\text{4FI}} ,
\end{align}
where $\Gamma_{\text{SM}} = \SI{84.00 \pm 0.01}{\MeV}$ is the theoretical prediction of the Standard Model to the $Z^0$ boson width decay (see Ref.~\cite{Beringer:1900zz}).

Neglecting the weak anomalous magnetic moment $a_f^Z$, the contribution due radiative correction at one-loop in this effective theory coming from RS setup with torsion is
\begin{align}
  \delta\Gamma_{\text{4FI}} &= -\frac{\alpha M_Z}{6\,s_Wc_W}\Bigg[(1 - 4s_W^2)\delta F_V^Z(M_Z^2) + \delta F_A^Z(M_Z^2)\Bigg].
\end{align}
Considering electron and positron in the final state, one obtain % and evaluating $\alpha$, vector and axial form factors at $Z^0$ pole one obtain 
\begin{align}
  \label{deltagammateo}
  \delta\Gamma_{\text{4FI}} &= \SI{9.87e6}{\MeV}\;\left(\frac{\si{\GeV}}{\Lambda}\right)^2\,\ln\left(\frac{\Lambda^2}{M_Z^2}\right).
\end{align}

The best updated experimental values of $Z^0$ boson width decay into electron-positron is (see  Ref.~\cite{Beringer:1900zz})
\begin{align}
  \label{deltagammaexp}
  \Gamma_{\text{exp}} &= \SI{83.984 \pm 0.086}{\MeV} ,
\end{align}
and the effects of beyond the standard model physics has to be included in the difference between the theoretical value and the experimental one. This leads to the equation 
\begin{align}
  \Gamma_{\text{th}} - \Gamma_{\text{exp}}\geq \delta\Gamma_{\text{4FI}}.
\end{align}
%% $. Restricting only to positive energy scales, and neglecting the scales excluded by the experiments (see Ref.~\cite{Chatrchyan:2013muj}), the  equation has solution for
%% \begin{eqnarray}
%%   \Lambda \geq \SI{35}{\TeV}\quad \text{at 95\%C.L.}
%% \end{eqnarray}
%% %at $95\%$C.L.

Due to the agreement between the theoretical and experimental quantities, the bound on $\Lambda$ rise up to \mbox{$\Lambda \ge \SI{70}{\TeV}$} at 95\% C.L., if only the mean values are considered. However,  the uncertainty in the experimental measurement induces a shift in the limits. Accordingly, the lower bound imposed by the width decay of the $Z^0$ boson drops down to
\begin{align}
  \Lambda &\ge \SI{24.5}{\TeV}\quad \text{at 95\%C.L.}\\
  \intertext{or}
  \Lambda &\ge \SI{27.3}{\TeV}\quad \text{at 95\%C.L.},
\end{align}
depending on the error in the theoretical prediction.

%The following figure shows how the contribution due four-fermion interaction coming from torsion to $Z^0$ boson width decay behaves depending on $\Lambda$. The blue curve line denotes $\delta\Gamma_{\text{4FI}}$ coming from torsionful extra dimension theory and the red straight one denotes $\delta\Gamma_{\text{exp}}$ obtained in SM precision tests.
The figure below shows how the width decay of the $Z^0$ boson behaves depending on $\Lambda$. % introduced through the four-fermion interaction induced by torsion.
The solid curve shows the variation of the width decay, $\delta\Gamma_{\text{4FI}}$, while the straight lines depict the allowed parameter region. The solid straight line is the more stringent limit, imposed by difference between the theoretical and experimental mean values of \mbox{$\Gamma(Z\to e^+\,e^-)$}. The dotted and dashed denote the shift of the allowed parameter region due to the experimental and theoretical errors obtained in SM precision tests.
\vspace{0.4cm}

\begin{tikzpicture}[scale=.9]
  \begin{axis}[
      title = Variation of $Z$ Width Decay,
      axis background/.style={
        shade,top color=gray!50,bottom color=white},
      xlabel = {$\Lambda$},
      ylabel = {$\delta\Gamma_{\text{4FI}}$},
      use units,
      x unit = {GeV}, y unit = {MeV},
      xmode = log,
      xmin = 8e1, xmax = 1e6,
    ]
    \addplot[ultra thick,
      dashed,
      red,
      samples = 301,
      domain=91.19:2e6,
    ] { 3.41e-9 };
    \addplot[ultra thick,
      blue,
      samples = 301,
      domain=91.19:2e6,
    ] { ln(x^2/(91.1876)^2)/x^2 };
    \coordinate (pt) at (axis cs: 3e4,1e-6);
  \end{axis}
  
  \node[pin=90:{
      \begin{tikzpicture}[baseline,trim axis left,trim axis right]
        \begin{axis}[
            axis background/.style={fill=white},
            footnotesize,
            grid = both,
            grid style = {black!20,dashed},
            %xmode  = log,% ymode = log,
            xmin = 1.8e4, xmax = 7.5e4,
            ymin = 1e-9, ymax = 4e-8,
          ]
          \addplot[thick,
            red,
            %name path=A,
            %% dashed,
            %% fill = red,
            %% fill opacity = .2,
            domain=1e4:9e4,
            samples=301,
          ] { 2.33e-9 }; %|- (axis cs:2000,-1e-9);
          
          \addplot[thick,
            red,
            dotted,
            fill = red,
            fill opacity = .15,
            domain=1e4:9e4,
            samples=301,
          ] { 1.86e-8 } |- (axis cs:2000,2.33e-9);
          
          \addplot[thick,
            red,
            dashed,
            fill = red,
            fill opacity = .15,
            domain=1e4:9e4,
            samples=301,
          ] { 1.53e-8 } |- (axis cs:2000,2.33e-9);
          
          \addplot[very thick,
            blue,
            domain=1e4:9e4,
            samples=301,
          ] { ln(x^2/(91.1876)^2)/x^2 };
        \end{axis}
      \end{tikzpicture}%
  }] at (pt) {};
\end{tikzpicture}



Comparing with the effective coupling coming from large extra dimension scenario, one find 
\begin{align}
  \frac{6}{32}\,\kappa_{\text{eff}}^2 \longleftrightarrow \frac{1}{\Lambda^2}.
\end{align}
To solve the Hierarchy Problem in RS scenario, the value $M_*\sim\SI{1}{\TeV}$ has been fixed. Using the stabilization value $k R \sim 10$ (see Ref.~\cite{Goldberger:1999uk}), one obtains the following limits on the compactification radius for different fermion localization values,
\begin{align}
  \label{rconst}
  R \lesssim 
  \begin{cases}
    \SI{e11}{m}; & c_i\simeq0 \\
    \SI{e7}{m}; & c_i\simeq1/2 \\
    \SI{e-14}{m}; & c_i\simeq1 
  \end{cases}
\end{align}
where the strongest limits comes from fermions localized near to the Planck brane, which was expected due the enhancement of the gravitational scale close to this brane.


\section{\label{sec:conclusions}Conclusions}

In this paper we have considered Dirac fermions coupled with the (minimal) Einstein-Cartan Theory of gravity. Within this framework, a four-fermion contact interaction rises. For this minimal generalization of the Einstein-Hilbert Theory of gravity, the new fermion interaction is suppressed by the gravitational scale, which in four dimensions is the Planck's mass, $M_{\text{Pl}}\sim\SI{e19}{\GeV}$.


In the present work a more general formulation of Gravity has been consider: the Cartan-Einstein Formalism. This framework gives rise to a contact four fermion interaction from the equations of motion, that is highly suppressed by the inverse of the squared Planck mass in four dimensions. 

In order to deal with this, an scenario without this hierarchy between gravitational and SM interactions has been used, RS model \cite{Randall:1999ee}. This model suggests the existence of one large extra dimension. However, in order to deal with $4$-dimensional observables, this large extra dimension is compactified on an orbifold $S^1/\mathbb{Z}_2$ of radius $R$. This compactification leads to an effective theory in $4$-dimensions with a four fermion interaction due torsion. The relation between the fundamental ($M_*$) and the effective ($M_{\text{Pl}}$) Planck mass, appears through the dimensional reduction as an exponential function.  

Considering this four fermion interaction in RS scenario, it can be decomposed in two terms: an axial-vector and axial-tensor interaction. By geometrical reasons of the present model this last term vanishes and leads only to the axial-vector one. One of the phenomenological implication of this absence is that, in RS model with one large extra dimension, torsion contributions to observables like leptonic anomalous magnetic moment are forbidden. 

Although this axial-tensor interaction does not appears in the effective theory one can explore axial-vector mediated observables as $Z^0$ boson width decay at one-loop level. In Sec.~\ref{sec:oneloop} form factors are obtained from neutral bosons exchange processes. Using these form factors and the calculation of $Z^0$ boson width decay, the scale for new physics coming from torsionful extra dimension scenario has been achieved. The use of SM precision tests data leads to a scale for new physics $ \Lambda \geq \SI{35}{\TeV}$ at $95\%$C.L., which strongly constraints the compactification radius of the extra dimension, for different fermion localization values, as Eq.~\eqref{rconst} shown.




\section*{Acknowledgement}

We would like to thank A. Toloza, V. Lyubovitskij for fruitful discussions. This work was supported by Conicyt (Chile) under Grant No. 21130179 and Fondecyt (Chile) under Grants No. 1100582, No. 1100287.




\bibliographystyle{apsrev4-1}
\bibliography{bibliography}




\end{document}
