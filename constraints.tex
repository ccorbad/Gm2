\section{Constraints from precision tests}
In this section, our aim is to compare the theoretical results obtained in the last section with values from precision tests. From these values, one could constraints the scale for new physics coming from extra dimensions scenarios.

\subsection{$Z^0$ boson width decay}

One of the most well known value, with excelent statistics and precision measurement is the $Z^0$ width decay. Considering the most general current $J^\mu(p,p')$ in Eq.~\eqref{current}, the width decay of $Z^0$ into electrons can be decomposed
\begin{align}
\Gamma_{\text{teo}}\left(Z^0\rightarrow e^+\,e^-\right) &= \Gamma_{\text{tree}} + \delta\Gamma_{1\text{-loop}}
\end{align}
where
\begin{align}
 \Gamma_{\text{tree}} &= \alpha M_Z\left[\frac{a_e^Z}{2s_Wc_W}(T_e^f - 2Q_f\,s_W^2)\right] 
\end{align}
is the the tree level contribution $Z^0$ boson width decay and
\begin{widetext}
\begin{align}
\nonumber
  \delta\Gamma_{1\text{-loop}} &= \frac{\alpha M_Z}{3\,s_Wc_W}\left[(T_3^f - 2Q_f\,s_W^2+s_wc_wa_e^Z)\delta F_V^Z(M_Z^2) + T_3^f\delta F_A^Z(M_Z^2) + \left(\frac{1}{2}(T_3^f - 2Q_f\,s_W^2) +\frac{M_Z^2}{4m_e^2}a_e^Z\right)\delta F_M^Z(M_Z^2)\right]
\end{align}
\end{widetext}
is the contribution due radiative correction at one loop. 

Considering electron and positron in the final state and evaluating form factors and $\alpha$ at $Z^0$ pole, one obtain 
\begin{align}
\label{deltagammateo}
 \delta\Gamma_{1\text{-loop}} &= -1.79777\times10^8\,[\text{GeV}]\left(\frac{[\text{GeV}]}{\Lambda}\right)^2\,\ln\left(\frac{\Lambda^2}{\mu^2}\right).
\end{align}

The best updated values of $Z^0$ boson width decay into electron positron come from Ref.~\cite{Beringer:1900zz}
\begin{align}
\label{deltagammaexp}
 \Gamma_{\text{exp}} &= 83.984 \pm 0.086\;[\text{MeV}] \equiv \Gamma_{\text{S.M.}} + \delta\Gamma_{\text{exp}}
\end{align}
and the effects of physics Beyond The Standard Model (BSM) has to be included in $\delta\Gamma_{\text{exp}}$. Choosing $\mu = m_t$ (top quark mass) as the major scale involved in the process one can compare Eq.~\eqref{deltagammateo} with Eq.~\eqref{deltagammaexp} and obtain the transcendental equation for the scale $\Lambda$ of physics BSM 
\begin{align}
\delta\Gamma_{\text{exp}} \geq \delta\Gamma_{\text{teo}}
\end{align}
that has solutions
\begin{eqnarray}
 \label{lowscalefromz}
 \Lambda_1 \leq&\; \pm0.17307\;&[\text{TeV}], \\
 \label{upscalefromz}
 \Lambda_2 \geq&\; \pm6.64309\times10^3\;&[\text{TeV}].
\end{eqnarray}

\subsection{Electron $g-2$}

Electron $g-2$ is also a well-known observable with high precision measurement and is related with magnetic form factor via
\begin{align}
\label{gm2}
\delta F_M^\gamma(k^2=0) &= a_e^\gamma \equiv \frac{1}{2}\left(g_e-2\right)
\end{align}
where $g_e$ is the anomalous magnetic moment of the electron. Considering SM and torsion contributions, Eq.~\eqref{gm2} can be decomposed
\begin{align}
a_e^\gamma &= a_{\text{SM}}^\gamma + \delta a_e^\gamma
\end{align}
where $\delta F_M^\gamma(k^2=0) = \delta a_e^\gamma$ denotes the contribution due radiative correction involving theory with four fermion interaction due torsion. 

Considering electrons in the t-channel of Eq.~\eqref{feyndiagram} with all possible fermions running into the loop, the contribution due torsion to $a_e^\gamma$ gives
\begin{align}
 \delta a_e^\gamma &= 0.22041\,\left(\frac{\text{GeV}}{\Lambda}\right)^2\,\ln\left(\frac{\Lambda^2}{\mu^2}\right).
\end{align}

The best updated data of electron $g-2$ are obtained in Ref.~\cite{Hanneke:2008tm} and also in Ref.~\cite{Beringer:1900zz}
\begin{align}
 a_e^\gamma &= (1159.65218076\pm0.00000027)\times10^{-6}
\end{align}

