\section{\label{sec:extradim}Extra dimensions scenario}


\subsection{Motivations}

In the previous section, a four fermion interaction due torsion has been found. One  problem with this interaction is that if one consider a four-dimensional scenario, the coupling constant $\kappa^2$ is suppressed by the inverse of the squared Planck mass. 

There exist extra dimensions models where the four-dimensional Planck scale $M_{\text{Pl}}$ is an exponential enhancement of a fundamental gravity scale $M_*$ defined on the whole manifold. Among these models one counts those known as %proposed by Arkani-Hamed, Dimopoulos and Dvali (
ADD models proposed  in Ref.~\cite{ArkaniHamed:1998rs,Antoniadis:1998ig,ArkaniHamed:1998nn}, %that consider $n\geq2$ compact extra dimensions 
and also the  Randall--Sundrum (RS) models proposed in Ref.~\cite{Randall:1999ee,Randall:1999vf} % that assumes $n=1$ large extra dimensions. 
with their generalizations  considered in Ref.~\cite{DeWolfe:1999cp,Gremm:1999pj,MPS,CastilloFelisola:2004eg}

In theories  with extra dimensions, one has to decompose the higher dimensional spinors, $\Psi$, into four-dimensional ones 
\begin{align}
  \label{spinordecomp}
  \Psi(x,y) &= N\,\sum_{i}\; \psi^{i}(x)\chi_{i}(y)
\end{align}
where $\psi$ and $x$ represent the four-dimensional spinor and coordinate system respectively, while $\chi$ and $y$ denote  $n$-dimensional ones, and $N$ is a normalization factor. In order to get an effective theory in four-dimensions, one has to perform dimensional reduction of the $n$ extra dimensions.

\subsection{The model}
In the following the  RS metric  will be consider~\cite{Randall:1999ee}
\begin{align}
  \label{RSmetric}
  ds^2 = e^{-2k\abs{y}}\eta_{\mu\nu}dx^\mu\,dx^\nu + dy^2.
\end{align}

In this scenario, the fifth dimension $y$ is compactified on an orbifold, $S^1/\mathbb{Z}_2$ of radius $R$ in the interval $0\leq y\leq \pi R$. One can define the five-dimensional vielbein from Eq.~\eqref{RSmetric} 
\begin{align}
  \vifhn{a}{} \equiv \left(\vifh^{a},\vifh^5\right) = \left(e^{-k|y|}\,dx^\mu,dy\right). 
\end{align}

Using
\begin{align}
|\hat{e}| = \det{\vih^{\hat{a}}_{\hat{\mu}}(x)} = e^{-4k|y|} 
\end{align}
the invariant volume element is %can be calculated using 
\begin{align}
   \dv[5]  = d^4x\, dy\;e^{-4k|y|}.
\end{align}
Note that the determinant of the vielbein has only dependence on the fifth dimension and can be integrated out when dimensional reduction is performed. 

The Kaluza-Klein (KK) decomposition for gauge and fermionic fields respectively are
\begin{align}
  \label{KKgaugedecomp}
  \hat{V}_{\hat{\mu}}^A(x,y) &= \frac{1}{\sqrt{2\pi R}}\sum_{i}h_{(i)}(y)V_{(i)\,\hat{\mu}}^{A}(x), \\
  \label{KKspindecomp}
  \Psi(x,y)_r &= \frac{1}{\sqrt{2\pi R}}\sum_{i}f_{(i)\,r}(y)\psi_{(i)\,r}(x),
\end{align}
where $R$ denotes the typical radius of the extra dimension, and the initial term is introduce as a normalization factor. On gauge boson, $A$ denotes Lie algebra index associated to the gauge symmetry. The extra dimension information of $\Psi_r$ and $\hat{V}_{\hat{\mu}}^A$ are enconded by $h_{(i)}(y)$ and $f_{(i)\,r}(y)$ profiles. The terms $\psi_{(i)\,r}(x)$ and $V_{(i)\,\hat{\mu}}^{A}(x)$ denotes fermion of $r$ flavor and gauge fields on four-dimensional spacetime, expanded on KK modes. 

In the previous KK decomposition for gauge fields, $\hat{V}_y^A=0$ has been used. This gauge choice eliminates $\hat{V}_y^A$ from the 3-brane but the gauge invariance of the effective action in four dimensions still remains (see Ref.~\cite{Davoudiasl:1999tf}). 

In the following analysis, only the zero mode of KK gauge and fermionic excitations $h_{(0)}(y)\equiv h(y)$ and $f_{(0)\,r}(y)\equiv f_r(y)$ respectively, will be consider. This approach gives only the lower mass of KK tower and allows to search in the threshold of finding extra dimensions, because the upper modes with higher masses will be less accessible. Using the zero mode of KK excitations obtained in Ref.~\cite{Gherghetta:2000qt,Gherghetta:2006ha} and the notation $V_{(0)\,\mu}^{A}(x)\equiv V_{\mu}^{A}(x)$ and $\psi_{(0)\,r}(x)\equiv\psi_r(x)$, the gauge and spinor fields can respectively be written
\begin{align}
 \hat{V}_{\mu}^A(x,y) &= \frac{1}{\sqrt{2\pi R}}V_\mu^A(x) \\
 \Psi_r(x,y) &= \frac{e^{(1/2-c_r)k|y|}}{N_r\sqrt{2\pi R}}\psi_r(x)
\end{align}
where $N_r$ is a normalization factor defined
\begin{align}
 N_r^2 = \frac{e^{2\pi kR(1/2-c_r)}-1}{2\pi kR(1/2 - c_r)}.
\end{align}
where $c_i$'s control the localization of fermions. % to Planck (UV) or TeV (IR) branes. If 
For $c_i>\tfrac{1}{2}$ %(<1/2)$, 
they are localized near to Planck brane,
for $c_i<\tfrac{1}{2}$ 
they are localized near to TeV brane,
while for $c=\tfrac{1}{2}$ fermions lie on the bulk.%a constant profile is obtained.


\subsection{Effective theory in four dimensions}

Clifford algebra in five dimensions can be constructed using the four-dimensional one. % plus the chiral gamma matrix defined by $\gamma^* = i\,\gamma^0\gamma^1\gamma^2\gamma^3$. Using tangent space coordinates, g
Gamma matrices in five-dimensional spacetimes are
\begin{align}
  \ghu{a} = \left(\gamma^a,\gamma^*\right).
\end{align}
With this definition the product of gamma matrices in Eq.~\eqref{4FI}, %using a general five-dimensional spacetime,
gives (See Appendix~\ref{sec:cliff}). 
\begin{align}
  (\ghhhu{a}{b}{c})(\ghhhd{a}{b}{c}) &= (\gamma^{abc})(\gamma_{abc}) + 3(\gamma^{ab*})(\gamma_{ab*}) \\
  &= 6\left(\gamma_a\gamma^*\right)\left(\gamma^a\gamma^*\right) + 3\left(\gamma^{ab}\gamma^*\right)\left(\gamma_{ab}\gamma^*\right)
\end{align}


If one consider only neutral current in the bulk as in Ref.~\cite{Davoudiasl:1999tf}, the interaction reads
\begin{align}
\label{gauge5D}
  S_{\text{int}} &= -i\sum_{r}\,g_5\int \dv[5]\,\bar{\Psi}_r \gamma^{\mu}T^A\Psi_r\hat{V}_\mu^A
\end{align}
where $g_5$ and $T^A$ is the gauge coupling coming from the fifth dimension and the generators of the Lie algebra associated to the gauge group respectively. In Eq.~\eqref{gauge5D} the identity
\begin{align}
\vih^{\hat{a}}_{\hat{\mu}}\;\hat{E}_{\hat{a}}^{\hat{\nu}} = \delta_{\hat{\mu}}^{\hat{\nu}}
\end{align}
has been used. Considering only the contribution due zero KK modes this interaction reads 
\begin{align}
  S_{\text{int}} \approx -i \sum_{r}\,g_{\text{eff}}\,\int d^4x\,\bar{\psi}_r\gamma^\mu T^A\psi_r V_\mu^A
\end{align}
where the effective coupling
\begin{align}
  g_{\text{eff}} \equiv \frac{g_5}{k(3+2c_r)}\,\frac{1-e^{-(3+2c_r)\pi kR}}{N_r^2(2\pi R)^{3/2}}
\end{align}
contains all the information of the gauge coupling in the bulk.

Similarly, the four fermionic interaction in Eq.~\eqref{4FI} coming from torsion, can be written as
\begin{align}
  \nonumber
  S_{4\text{FI}} &\approx \sum_{r,s}\,\frac{\kappa_{\text{eff}}^2}{32}\,\int d^4x\left(6\,\bar{\psi}_{r}\gamma^\mu\gamma^*\psi_{r}\,\bar{\psi}_{s}\gamma_\mu\gamma^*\psi_{s}\right.\\
  \label{4FI5D}
  &+\left. 3\,\bar{\psi}_r\gamma^{\mu\nu}\gamma^*\psi_r\,\bar{\psi}_s\gamma_{\mu\nu}\gamma^*\psi_s\right)
\end{align}
where $\kappa_{\text{eff}}^2$ has been defined in terms of the zero modes of KK excitations giving
\begin{align}
\label{kapparel}
  k_{\text{eff}}^2 \equiv \frac{1-e^{-2(1+c_r+c_s)\pi kR}}{8k\,N_r^2N_s^2(1+c_r+c_s)(\pi R)^2}\,\kappa_*^2.
\end{align}

An extra axial-tensor term arises by considering only one extra dimension in Eq.~\eqref{4FI5D}. If one wants chiral fermions in the effective theory in $4$-dimensions the orbifold boundary condition $\pm\gamma^*f_r(y)=f_r(-y)$ must be satisfied, as Ref.~\cite{Flachi:2001bj} shown. Analysing the last term on Eq.~\eqref{4FI5D} one can verify that, before the dimensional reduction, the term $\bar{\Psi}_r\gamma^{\mu\nu}\gamma^*\Psi_r$ is odd under $y\rightarrow-y$. This leads to a trivial contribution of axial-tensor current interaction and torsion four fermion interaction is decomposed into purely axial-vector one (see Ref.~\cite{Lebedev:2002dp}).

One can consider the stabilization value $kR\sim10$ from Ref.~\cite{Goldberger:1999uk} and differents values $c_i$
\begin{align}
\kappa_{\text{eff}}^2 &= \left\{ \begin{matrix}
                           2.8388\times10^{-15}\,\kappa_*^2\,k & \text{if} & c_r\simeq c_s\simeq 0 \\
                           6.3325\times10^{-5}\,\kappa_*^2\,k & \text{if} & c_r\simeq c_s\simeq 1/2 \\
                           4.1666\times10^{-2}\,\kappa_*^2\,k & \text{if} & c_r\simeq c_s\simeq 1 
                          \end{matrix} \right.
\end{align}

In order to analyse only the fundamental scale of extra dimensions, the effective coupling of neutral gauge bosons coupled to fermionic current, will be considered equal as the SM in $4D$, that is, $g_{\text{eff}} = e\,Q_f$ for photon exchange and $g_{\text{eff}} = e/(2c_Ws_W)\left(T^3_f - 2\,s_W^2\,Q_f\right)$ for $Z^0$ exchange, where $Q_f$ is electric charge in proton unities, $T_f^3$ is the third component of weak isospin and $s_W(c_W) = \sin(\cos)\theta_W$. 




